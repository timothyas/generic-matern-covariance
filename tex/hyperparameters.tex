\section{Parameterizing the Correlation Model}
\label{sec:matern_operator}

Here we propose to use the SPDE operator described by \citet{RSSB:RSSB777}
together with the correlation/covariance formulation suggested by WC01.
That is, we define a correlation operator via its square root as
\begin{linenomath*}\begin{equation}
    \corrMat^{1/2} \coloneqq \normalizer
    \Big(\delta(\x)- \nabla\cdot K(\x)\nabla\Big)^{-1}
    \defdet^{-1/2}
\end{equation}\end{linenomath*}
where $\corrMat \coloneqq \corrMat^{1/2}\corrMat^{T/2}$, and $\normalizer$ is
once again a variance-preserving normalization matrix defined by the operations
that precede it.
The degree to which neighboring points are correlated are controlled by
$\defjac(\x)$ and $\deltah$, and here we discuss how these can be assigned for
practical applications in oceanographic state estimation.
We note that in this discussion we loosely use the infinite dimensional forms of
these operators in order to ease the presentation, but provide a more careful
derivation of their discretized form in \cref{sec:discretization_matern}.

\subsection{Scaling the Laplacian Term}
\label{ssec:scaling_laplacian}

First, we address the Jacobian, $\defjac(\x)$, which appears in the tensor
$K(\x)$.
We illustrate how this operator controls correlation length scales and
motivate its parameterization with a simple scaling analysis.
Consider a 3D field $\uni(\x)\sim U$ that exhibits spatial variability at the
length scales, $L_h$, $L_h$, and $L_z$ in each dimension
where $L_h, L_h >> L_z$, such that the field exhibits highly
anisotropic fluctuations.
This is a common situation in large scale geophysical fluid
dynamics, where fields (e.g.\ temperature, velocity) exhibit length scales of
variability that are much greater in either horizontal dimension compared to the
vertical.
\footnote{
    We make a note on the terminology used here.
    In oceanography, the difference in horizontal and vertical scales
    is a result of the small aspect ratio, or the shallow fluid nature of the ocean.
    Because of the vast difference in scales, the horizontal dimensions are usually
    considered entirely independent of the vertical, and ``anisotropy'' is typically
    used to refer to heterogeneity between the horizontal components.
    However, here we use anisotropy to refer to the difference between horizontal
    and vertical scales, resulting from the small aspect ratio \citep{vallis2006}.
}

Without any rescaling, i.e.\ without $K$,
the Laplacian term in $\maternOp$ is imbalanced
\begin{linenomath*}\begin{equation}
    \begin{aligned}
        \nabla^2 \uni(\x)
            & \sim \dfrac{U}{L_h^2} + \dfrac{U}{L_h^2} + \dfrac{U}{L_z^2} \\
            & \simeq \dfrac{U}{L_z^2} \, .
    \end{aligned}
    \label{eq:iso_lap}
\end{equation}\end{linenomath*}
As a result, the correlation model will
have unrealistically large or small correlations in the horizontal or vertical.
Our goal is therefore to define the elements of $K$ such that each term is of
the same order of magnitude and
\begin{linenomath*}\begin{equation*}
    \nabla\cdot K\nabla \sim 3U \, .
\end{equation*}\end{linenomath*}

To achieve this balance between Laplacian terms, we suggest a straightforward,
perhaps obvious, specification of $\defjac$:
\begin{linenomath*}\begin{equation*}
    \defjac =
        \begin{pmatrix}
            L_h & 0 & 0     \\
            0 & L_h & 0     \\
            0 & 0   & L_z   \\
        \end{pmatrix} \, ,
\end{equation*}\end{linenomath*}
where we simply ignore the off-diagonal elements of $\defjac$.
The determinant in this case is $\defdetnox = L_h^2L_z$ and
according to the definitions in equation (\ref{eq:matern_definitions}):
\begin{linenomath*}\begin{equation*}
    K =
        \begin{pmatrix}
            1/L_z & 0 & 0     \\
            0 & 1/L_z & 0     \\
            0 & 0   & L_z/L_h^2   \\
        \end{pmatrix} \, ,
\end{equation*}\end{linenomath*}
so that
\begin{linenomath*}\begin{equation*}
    \nabla\cdot K(\x)\nabla\uni(\x) \sim \dfrac{3}{L_h^2L_z}U \, .
\end{equation*}\end{linenomath*}
The key is that $K$ scales each term in the Laplacian so that they are
approximately the same order of magnitude, and the operator is balanced in either direction.
%This example can further be extended by allowing the length scales $L_1$ and
%$L_2$ to vary in space, such that correlations are nonstationary.

At this point, we must prescribe values for $L_h(\x)$ and $L_z(\x)$ to fill
$\defjac(\x)$.
Considering the discretization of the Laplacian, a simple choice for these can
be based on the underlying grid of the ocean model.
For the numerical experiments in this paper, we choose
\red{$L_h(i,j) = \sqrt{\Delta x_g(i,j) \Delta y_g(i,j)}$},
and $L_z(k) = \Delta r_f(k)$ (\cref{fig:mitgcm_grid}) where $i$, $j$, $k$ refer
to indices of the computational grid.
We consider the choice to use the grid elements directly to be reasonable
because anisotropy and nonstationarity is usually encoded into the grid.
For instance, regarding nonstationarity, it is reasonable to assume that
correlation length scales near the surface of the ocean are relatively short as
a result of the locality of atmosphere-ocean interactions that occur there.
On the other hand, near the ocean floor motion is presumed to be more
quiescient, and correlation length scales are correspondingly longer.
Many ocean models
\citep[e.g.][]{nguyen_arctic_2021, forgetECCOv4}
incorporate these assumptions into the vertical grid resolution.


\subsection{The Range Parameter}
\label{ssec:range_parameter}

The other term in (\cref{eq:spde_general}) to be defined is $\deltah$.
Here we resort to the empirical relation suggested in
\citet{RSSB:RSSB777}:
\begin{linenomath*}\begin{equation*}
    \rangeh = \sqrt{\dfrac{8\meandiff}{\deltah}} \, .
\end{equation*}\end{linenomath*}
The so-called range parameter, $\rangeh$, defines the
distance between two points at which correlation drops to 0.1.
While this may seem like swapping one unknown for the other, it is usually easier to
define correlation length scales than simply guessing values for $\deltah$.
With this relation and our choice for $\defjac(\x)$ in
\cref{ssec:scaling_laplacian} we have
\begin{linenomath*}\begin{equation*}
    \begin{aligned}
        \delta(i,j,k) &= \dfrac{8\meandiff}{\rangeh^2\defdetd} \\
                 &= \dfrac{8\meandiff}{\rangeh^2 L_h(i,j,k)^2 L_z(i,j,k)} \, .
    \end{aligned}
\end{equation*}\end{linenomath*}
With this choice, both this term and the Laplacian term are now of a similar order of
magnitude
\begin{linenomath*}\begin{equation*}
    D_{\delta}\unis \sim \nabla \cdot K \nabla \unis \sim
    \bigo\left(\dfrac{1}{L_h^2 L_z}\right) \, ,
\end{equation*}\end{linenomath*}
where we note that $\meandiff \sim \bigo(1)$, and we take this to be
$\meandiff = 1/2$ for the experiments in this work since $d=3$, see
\cref{eq:spde_iso}.
With $L_h$ and $L_z$ defined by the model grid elements, one can intuitively
regard $\rangeh$ as a nondimensional parameter that controls the
``number of neighboring grid cells'' at which correlation decays to 0.1.
Our numerical experiments in \cref{sec:llc90} make this clear.
