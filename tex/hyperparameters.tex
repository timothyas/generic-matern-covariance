\section{A Nonstationary and Anisotropic Mat\'ern Correlation Operator}
\label{sec:matern_operator}

Here we propose to use the SPDE operator described by \citetalias{RSSB:RSSB777}
as a means to describe an anisotropic, nonstationary correlation model in a
similar manner to the diffusion-based methods described in
\cref{ssec:wc01_review}.
To do so, we employ the ``mapping method'' described by \citetalias{RSSB:RSSB777}
which we show for the general $M$th order SPDE in \cref{ssec:mapping_method}.
In summary, the basic idea we present is to use the mapping method to
nondimensionalize or re-scale the elliptic operator.
We provide a simple scaling argument for why this is a good idea in
\cref{ssec:scaling_laplacian}, and discuss practical choices for the
nondimensionalization based on the grid scale in \cref{ssec:nonstationarity}.

\subsection{Mapping method or change of variables}
\label{ssec:mapping_method}

In \citetalias{RSSB:RSSB777} it is suggested that solving the SPDE in a transformed
coordinate system allows one to readily incorporate anisotropy and
nonstationarity into a Mat\'ern covariance model.
In this mapping method, we consider solutions to the isotropic, stationary SPDE
(\ref{eq:spde_iso}) to be defined in a transformed, or
``deformed'' \citep{sampson_nonparametric_1992}, space $\defdomain$.
Then, assume that we have a mapping $\defmap$ that maps between this transformed space
and our computational domain, $\domain$:
\begin{linenomath*}\begin{equation*}
    \defmap : \defdomain\ni\xh \rightarrow \x\in\domain \, .
\end{equation*}\end{linenomath*}
With this mapping, we can employ a change of variables
\citep{smith_change_1934} to rewrite the SPDE in the computational domain as:
\begin{linenomath*}\begin{equation*}
    \left(\dfrac{\deltah}{\defdet} -
    \nabla\cdot
    \dfrac{\defjac(\x)\defjac(\x)^T}{\defdet}
    \nabla\right)^M\uni(\x) =
    \defdet^{-1/2}\W(\x) \, .
\end{equation*}\end{linenomath*}
Here we have defined the Jacobian as
\begin{linenomath*}\begin{equation*}
    \defjac(\x_0) \coloneqq
    \dfrac{\partial \defmap}{\partial \xh}\Big|_{\defmap^{-1}(\x_{0})} \, ,
\end{equation*}\end{linenomath*}
and for now we assume that $\defmap^{-1}(\x_0)$ is well defined.
For our purposes, this turns out to be the case, but this becomes clear when
$\defjac$ is defined in \cref{ssec:scaling_laplacian}.
With the following definitions:
\begin{linenomath*}\begin{equation}
        K(\x) \coloneqq
        \dfrac{\defjac(\x)\defjac(\x)^T}{\defdet}
        \qquad
        \delta(\x) \coloneqq \dfrac{\deltah}{\defdet}
        \qquad
        \maternop \coloneqq \left(\delta(\x) - \nabla\cdot K(\x)\nabla\right)
    \label{eq:matern_definitions}
\end{equation}\end{linenomath*}
the SPDE in \cref{eq:spde_iso} can be written in the computational domain's coordinate system as
\begin{linenomath*}\begin{equation}
    \maternop^M \uni(\x) =
    \defdet^{-1/2}\W(\x) \, ,
    \label{eq:spde_general}
\end{equation}\end{linenomath*}
where zero flux, Neumann boundary conditions are applied at the boundaries
(see \cref{sec:discretization_matern} for details).
%We note as in \citetalias{RSSB:RSSB777} that this reproduces the deformation method
%introduced in \citet{sampson_nonparametric_1992}.

Here, we propose to use this generic form to define a square root of the
correlation matrix in a similar fashion to
\citet{weaver_correlation_2001, mirouze_representation_2010,
carrier_background-error_2010}
as follows,
\begin{linenomath*}\begin{equation}
    \corrMat^{1/2} \coloneqq \normalizer
    \maternop^{-M}
    \defdet^{-1/2} \, ,
    \label{eq:matern_operator}
\end{equation}\end{linenomath*}
where $\normalizer$ is
once again a variance-preserving normalization matrix defined by the operations
that precede it.
In this model,
anisotropy and nonstationarity are controlled by
$\defjac(\x)$, and in the following sections
we discuss how this can be assigned for
practical applications in geophysical inverse problems.
We note that in this discussion we loosely mix the use of
finite dimensional matrices and infinite dimensional operators
in order to ease the presentation, but we provide a more careful
derivation of their discretized forms relevant to our numerical experiments in
\cref{sec:discretization_matern}.

\subsection{Scaling the Laplacian term for anisotropy}
\label{ssec:scaling_laplacian}

Here we focus on parameterizing $\defjac(\x)$ in order to achieve an anisotropic
correlation model that is relevant for variational DA.
We illustrate our choice with a scaling argument focusing on how $K(\x)$
influences correlation length scales.

Consider a 3D field $\uni(\x)\sim \uniscaling$ that exhibits spatial variability at the
length scales, $L_x$, $L_y$, and $L_z$ in the direction of longitude, latitude,
and height, respectively,
where $L_x$, $L_y$ $>>$ $L_z$, such that the field exhibits highly
anisotropic fluctuations.
This is a common situation in large scale geophysical fluid
dynamics, where fields (e.g., temperature, velocity) exhibit length scales of
variability that are much greater in either horizontal dimension compared to the
vertical \citep[e.g.,][]{vallis2006}.
%\footnote{
%    We make a note on the terminology used here.
%    In oceanography, the difference in horizontal and vertical scales
%    is a result of the small aspect ratio, or the shallow fluid nature of the ocean.
%    Because of the vast difference in scales, the horizontal dimensions are
%    sometimes considered entirely independent of the vertical, and ``anisotropy''
%    can be used to refer to heterogeneity between the horizontal components.
%    However, here we use anisotropy to refer to the difference between horizontal
%    and vertical scales, resulting from the small aspect ratio \citep{vallis2006}.
%}
Without any rescaling, i.e.\ without $K$,
the Laplacian term in $\maternop$ is unbalanced
\begin{linenomath*}\begin{equation}
    \begin{aligned}
        \nabla^2 \uni(\x)
            & \sim \dfrac{\uniscaling}{L_x^2} + \dfrac{\uniscaling}{L_y^2} + \dfrac{\uniscaling}{L_z^2} \\
            & \simeq \dfrac{\uniscaling}{L_z^2} \, .
    \end{aligned}
    \label{eq:iso_lap}
\end{equation}\end{linenomath*}
As a result, the correlation model will
have unrealistically large (small) correlations in the vertical (horizontal).
Our goal is therefore to define the elements of $K$ such that each term is of
the same order of magnitude.
%and
%\begin{linenomath*}\begin{equation*}
%    \nabla\cdot K\nabla \sim 3\uniscaling \, .
%\end{equation*}\end{linenomath*}

To achieve this balance between Laplacian terms, we suggest a straightforward,
perhaps obvious, specification of $\defjac$:
\begin{linenomath*}\begin{equation*}
    \defjac =
        \begin{pmatrix}
            L_x & 0 & 0     \\
            0 & L_y & 0     \\
            0 & 0   & L_z   \\
        \end{pmatrix} \, ,
\end{equation*}\end{linenomath*}
where we simply ignore the off-diagonal elements of $\defjac$.
The determinant in this case is $\defdetnox = L_xL_yL_z$ and
according to the definitions in \cref{eq:matern_definitions}:
\begin{linenomath*}\begin{equation*}
    K =
        \begin{pmatrix}
            1/L_z & 0 & 0     \\
            0 & 1/L_z & 0     \\
            0 & 0   & L_z/(L_xL_y)   \\
        \end{pmatrix} \, ,
\end{equation*}\end{linenomath*}
so that
\begin{linenomath*}\begin{equation*}
    \nabla\cdot K\nabla\uni(\x) \sim \dfrac{3}{L_xL_yL_z}\uniscaling \, .
\end{equation*}\end{linenomath*}
The key is that $K$ scales each term in the Laplacian so that they are
approximately the same order of magnitude, and the operator is balanced in all
directions.
In the case of nonstationarity, we simply require this balance to apply locally
and allow the length scales $L_x$, $L_y$, and $L_z$ to vary in space.
%This example can further be extended by allowing the length scales $L_1$ and
%$L_2$ to vary in space, such that correlations are nonstationary.


\subsection{Harnessing the grid scale for nonstationarity}
\label{ssec:nonstationarity}

At this point, we must prescribe values for the normalizing length scales
$L_x(\x)$, $L_y(\x)$, and $L_z(\x)$ in order to fill $\defjac(\x)$.
Considering the scaling analysis of the Laplacian in
\cref{ssec:scaling_laplacian}, a simple yet reliable choice for these
is to use the underlying grid-scale of the numerical model (e.g., the ocean or
atmosphere general circulation model).

We consider using the length scale of the grid elements to be reasonable
because a baseline level anisotropy and nonstationarity is usually encoded
into the grid.
A prime example of this nonstationarity is represented by the vertical axis of
ocean model grids, which are designed to capture a variety of behavior in a
computationally efficient manner \citep{griffies_fundamentals_2004}.
Toward the surface, the ocean is tightly coupled to the atmosphere, sea ice, and
rivers, and ocean models use a finely resolved vertical grid to capture the
ocean component of these coupled processes.
In the interior ocean, well below the mixed layer, the ocean acts more like stacked layers, and
variation in properties like temperature and salinity occurs over much larger
distances than at the surface \citep{talley_descriptive_2011}.
Vertical grids are correspondingly much coarser at depth than near the surface.
As a concrete example, the height-based vertical grid we use in \cref{sec:llc90}
varies from $\sim 5-10$~m near the surface, and spacing increases to
$\bigo(100)$~m below 1,000~m (\cref{fig:llc90_correlation_maps}(f)).
By using the grid elements directly, our correlation model can capture the
nonstationarity motivated by the physical processes which influence the model
grid's development.

Another justification for using the grid elements to specify $\defjac(\x)$ is
that this provides a practical and intuitive nondimensionalization.
With this definition, the correlation model exhibits isotropic and stationary
behavior within the nondimensional space defined by the computational grid.
As such, the idealized correlation function (\cref{eq:matern_corr_isostat})
applies relative to the grid spacing, and one can view $\rangeh$ as an
intuitive, nondimensional parameter controlling the
``number of neighboring grid cells'' at which correlation decays to 0.14.
Our numerical experiments in \cref{sec:llc90} show that this is a good
approximation in the case of a realistic global ocean model grid.
