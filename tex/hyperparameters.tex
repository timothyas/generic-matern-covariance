\section{Specification of covariance model hyperparameters}
\label{sec:matern_operator}

\noindent\textbf{Rescaling the Laplacian term.}
Consider a 2D spatial field $\uni(\x)$ that exhibits spatial variability at the
length scales $L_y$ and $L_z$ in each dimension.
We motivate our approach by considering the magnitude of
the Laplacian term without any rescaling (i.e. without $K$):
\begin{linenomath}\begin{equation}
    \nabla^2 \uni(\x) \sim \dfrac{U}{L_y^2} + \dfrac{U}{L_z^2}
    \label{eq:iso_lap}
\end{equation}\end{linenomath}
where we suppose that $\uni(\x)\sim U$.
We finally suppose that $L_y >> L_z$, such that the field experiences highly
anisotropic fluctuations.
This is a common situation in large scale geophysical fluid
dynamics, where fields (e.g.\ temperature, salinity, velocity) exhibit length scales of
variability that are much greater in either horizontal dimension compared to the
vertical.
We make a note on the terminology used here.
In oceanography, the difference in horizontal and vertical scales
is a result of the small aspect ratio, or the shallow fluid nature of the ocean.
Because of the vast difference in scales, the horizontal dimensions are usually
considered entirely independent of the vertical, and ``anisotropy'' is typically
used to refer to heterogeneity between the horizontal components.
However, here we use anisotropy to refer to the difference between horizontal
and vertical scales, resulting from the small aspect ratio \citep{vallis2006}.

The specification of an isotropic covariance model amounts to specifying scalar
parameters in equation (\ref{eq:spde_iso}).
In our hypothetical anisotropic  situation, this would result in unrealistically large or
small correlation lengths along either dimension.
Numerically this occurs because of the lack of balance between the terms in
equation (\ref{eq:iso_lap}): $U/L_y^2 << U/L_z^2$.
Our goal is therefore to define the elements of $K$ such that
\begin{linenomath*}\begin{equation*}
    \nabla\cdot K\nabla \sim (a+b)U
\end{equation*}\end{linenomath*}
where $a \sim b$ are of the same order of magnitude.

To achieve this balance between Laplacian terms, we suggest a straightforward,
perhaps obvious, specification of $\defjac$:
\begin{linenomath*}\begin{equation*}
    \defjac =
        \begin{pmatrix}
            L_y & 0     \\
            0   & L_z   \\
        \end{pmatrix} \, ,
\end{equation*}\end{linenomath*}
where we simply ignore the off-diagonal elements of $\defjac$.
The determinant in this case is $\defdetnox = L_yL_z$ and
according to the definitions in equation (\ref{eq:matern_definitions}):
\begin{linenomath*}\begin{equation*}
    K =
        \begin{pmatrix}
            L_y/L_z & 0     \\
            0   & L_z/L_y   \\
        \end{pmatrix} \, ,
\end{equation*}\end{linenomath*}
so that
\begin{linenomath*}\begin{equation*}
    \nabla\cdot K(\x)\nabla\uni(\x) \simeq L\unis  \sim \dfrac{2}{L_yL_z}U \, .
\end{equation*}\end{linenomath*}
The key is that $K$ scales each term in the Laplacian so that they are
approximately the same, and the operator is balanced in either direction.\\
%This example can further be extended by allowing the length scales $L_1$ and
%$L_2$ to vary in space, such that correlations are nonstationary.

\noindent\textbf{The range parameter.}
The other term in equation (\ref{eq:spde_general}) to be defined is $\deltah$.
Here we resort to the empirical relation suggested in
\citet{RSSB:RSSB777}:
\begin{linenomath*}\begin{equation*}
    \rangeh = \sqrt{\dfrac{8\meandiff}{\deltah}} \, .
\end{equation*}\end{linenomath*}
The so-called range parameter, $\rangeh$, defines the
distance between two points at which correlation drops to 0.1.
While this may seem like swapping one unknown for the other, it is usually easier to
define correlation length scales than simply guessing values for $\deltah$.
With this relation we have
\begin{linenomath*}\begin{equation*}
    \begin{aligned}
        \delta_i &= \dfrac{8\meandiff}{\rangeh^2\defdetdi} \\
                 &= \dfrac{8}{\rangeh^2 L_y L_z} \, ,
    \end{aligned}
\end{equation*}\end{linenomath*}
where we have substituted $\meandiff=1$, avoiding fractional or higher order exponents
in the SPDE (recall equation \eqref{eq:spde_iso}), and $\defdetdi = L_yL_z$.
Note that both this term and the Laplacian term are now of a similar order of
magnitude
\begin{linenomath*}\begin{equation*}
    D_{\delta}\unis \sim L\unis \sim \bigo\left(\dfrac{1}{L_y L_z}\right)
\end{equation*}\end{linenomath*}
\\

\noindent\textbf{Practical specification of the hyperparameters.}
With the definitions above, now one has to assign $L_y$, $L_z$, and $\rangeh$.
Considering the discretization of the Laplacian, a simple and intuitive choice
for $L_y$ and $L_z$ would be grid scale elements $\Delta y$ and $\Delta r$,
or factors thereof.
From a scaling analysis, this seems intuitive since
\begin{linenomath*}\begin{equation*}
    \begin{aligned}
        D_z
        &=
        \text{diag}\left\{\dfrac{1}{
            \sqrt{\vol_i\defdetdi}}\right\}_{i=1}^{\nuni} \\
        &=
        \text{diag}\left\{\dfrac{1}{
            \sqrt{\Delta y_{g}^{i}\Delta r_{f}^{i}
        \defdetdi}}\right\}_{i=1}^{\nuni} \\
        &= \text{diag}\left\{\dfrac{1}{
            \Delta y_{g}^{i}\Delta r_{f}^{i}}\right\}_{i=1}^{\nuni}
    \end{aligned}\, ,
\end{equation*}\end{linenomath*}
for instance if $L_y=\Delta y_g$ and $L_z=\Delta r_f$, so that
\begin{linenomath*}\begin{equation*}
    D_z\mathbf{z} \sim \bigo\left(\dfrac{1}{L_y L_z}\right) \, ,
\end{equation*}\end{linenomath*}
and all terms are of the same order of magnitude.
In general circulation models, these choices for $L_y$ and $L_z$ are generally
reasonable because anisotropy is encoded into the mesh.
It turns out that specifying the length scales as such provides for a
practical implementation, as the range parameter can be chosen as a simple
scaling to amplify or attenuate the correlation lengths.
One can then intuitively regard $\hat\rho$ as a nondimensional parameter that
controls the
``number of neighboring grid cells'' at which correlation decays to 0.1.
Our numerical experiments in section \ref{sec:matern_pig} make this clear.
