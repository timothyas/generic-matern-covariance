\section{Parameterizing the Correlation Model}
\label{sec:matern_operator}

Here we propose to use the SPDE operator described by \citet{RSSB:RSSB777}
together with the correlation/covariance formulation suggested by WC01.
That is, we define a correlation operator via its square root as
\begin{linenomath*}\begin{equation}
    \corrMat^{1/2} \coloneqq \normalizer
    \Big(\delta(\x)- \nabla\cdot K(\x)\nabla\Big)^{-1}
    \defdet^{-1/2}
\end{equation}\end{linenomath*}
where $\corrMat \coloneqq \corrMat^{1/2}\corrMat^{T/2}$, and $\normalizer$ is
once again a variance-preserving normalization matrix defined by the operations
that precede it.
The degree to which neighboring points are correlated are controlled by
$\defjac(\x)$ and $\deltah$, and here we discuss how these can be assigned for
practical applications in oceanographic state estimation.
We note that in this discussion we loosely use the infinite dimensional forms of
these operators in order to ease the presentation, but provide a more careful
derivation of their discretized form in \cref{sec:discretization_matern}.

First, we address the Jacobian $\defjac(\x)$, which appears in the tensor
$K(\x)$.
We illustrate how this operator controls correlation length scales and
motivate its parameterization with a simple scaling analysis.
Consider a 3D field $\uni(\x)\sim U$ that exhibits spatial variability at the
length scales, $L_h$, $L_h$, and $L_z$ in each dimension
where $L_h, L_h >> L_z$, such that the field exhibits highly
anisotropic fluctuations.
This is a common situation in large scale geophysical fluid
dynamics, where fields (e.g.\ temperature, salinity, velocity) exhibit length scales of
variability that are much greater in either horizontal dimension compared to the
vertical.
\footnote{
    We make a note on the terminology used here.
    In oceanography, the difference in horizontal and vertical scales
    is a result of the small aspect ratio, or the shallow fluid nature of the ocean.
    Because of the vast difference in scales, the horizontal dimensions are usually
    considered entirely independent of the vertical, and ``anisotropy'' is typically
    used to refer to heterogeneity between the horizontal components.
    However, here we use anisotropy to refer to the difference between horizontal
    and vertical scales, resulting from the small aspect ratio \citep{vallis2006}.
}

Without any rescaling, i.e.\ without $K$,
the Laplacian term in $\maternOp$ is imbalanced
\begin{linenomath*}\begin{equation}
    \begin{aligned}
        \nabla^2 \uni(\x)
            & \sim \dfrac{U}{L_h^2} + \dfrac{U}{L_h^2} + \dfrac{U}{L_z^2} \\
            & \simeq \dfrac{U}{L_z^2} \, .
    \end{aligned}
    \label{eq:iso_lap}
\end{equation}\end{linenomath*}
As a result, the correlation model will
have unrealistically large or small correlations in the horizontal or vertical.
Our goal is therefore to define the elements of $K$ such that each term is of
the same order of magnitude and
\begin{linenomath*}\begin{equation*}
    \nabla\cdot K\nabla \sim 3U \, .
\end{equation*}\end{linenomath*}

To achieve this balance between Laplacian terms, we suggest a straightforward,
perhaps obvious, specification of $\defjac$:
\begin{linenomath*}\begin{equation*}
    \defjac =
        \begin{pmatrix}
            L_h & 0 & 0     \\
            0 & L_h & 0     \\
            0 & 0   & L_z   \\
        \end{pmatrix} \, ,
\end{equation*}\end{linenomath*}
where we simply ignore the off-diagonal elements of $\defjac$.
The determinant in this case is $\defdetnox = L_h^2L_z$ and
according to the definitions in equation (\ref{eq:matern_definitions}):
\begin{linenomath*}\begin{equation*}
    K =
        \begin{pmatrix}
            1/L_z & 0 & 0     \\
            0 & 1/L_z & 0     \\
            0 & 0   & L_z/L_h^2   \\
        \end{pmatrix} \, ,
\end{equation*}\end{linenomath*}
so that
\begin{linenomath*}\begin{equation*}
    \nabla\cdot K(\x)\nabla\uni(\x) \sim \dfrac{3}{L_h^2L_z}U \, .
\end{equation*}\end{linenomath*}
The key is that $K$ scales each term in the Laplacian so that they are
approximately the same order of magnitude, and the operator is balanced in either direction.\\
%This example can further be extended by allowing the length scales $L_1$ and
%$L_2$ to vary in space, such that correlations are nonstationary.

\noindent\textbf{The range parameter.}
The other term in equation (\ref{eq:spde_general}) to be defined is $\deltah$.
Here we resort to the empirical relation suggested in
\citet{RSSB:RSSB777}:
\begin{linenomath*}\begin{equation*}
    \rangeh = \sqrt{\dfrac{8\meandiff}{\deltah}} \, .
\end{equation*}\end{linenomath*}
The so-called range parameter, $\rangeh$, defines the
distance between two points at which correlation drops to 0.1.
While this may seem like swapping one unknown for the other, it is usually easier to
define correlation length scales than simply guessing values for $\deltah$.
With this relation we have
\begin{linenomath*}\begin{equation*}
    \begin{aligned}
        \delta_i &= \dfrac{8\meandiff}{\rangeh^2\defdetdi} \\
                 &= \dfrac{8}{\rangeh^2 L_y L_z} \, ,
    \end{aligned}
\end{equation*}\end{linenomath*}
where we have substituted $\meandiff=1$, avoiding fractional or higher order exponents
in the SPDE (recall equation \eqref{eq:spde_iso}), and $\defdetdi = L_yL_z$.
Note that both this term and the Laplacian term are now of a similar order of
magnitude
\begin{linenomath*}\begin{equation*}
    D_{\delta}\unis \sim L\unis \sim \bigo\left(\dfrac{1}{L_y L_z}\right)
\end{equation*}\end{linenomath*}
\\

\noindent\textbf{Practical specification of the hyperparameters.}
With the definitions above, now one has to assign $L_y$, $L_z$, and $\rangeh$.
Considering the discretization of the Laplacian, a simple and intuitive choice
for $L_y$ and $L_z$ would be grid scale elements $\Delta y$ and $\Delta r$,
or factors thereof.
From a scaling analysis, this seems intuitive since
\begin{linenomath*}\begin{equation*}
    \begin{aligned}
        D_z
        &=
        \text{diag}\left\{\dfrac{1}{
            \sqrt{\vol_i\defdetdi}}\right\}_{i=1}^{\nuni} \\
        &=
        \text{diag}\left\{\dfrac{1}{
            \sqrt{\Delta y_{g}^{i}\Delta r_{f}^{i}
        \defdetdi}}\right\}_{i=1}^{\nuni} \\
        &= \text{diag}\left\{\dfrac{1}{
            \Delta y_{g}^{i}\Delta r_{f}^{i}}\right\}_{i=1}^{\nuni}
    \end{aligned}\, ,
\end{equation*}\end{linenomath*}
for instance if $L_y=\Delta y_g$ and $L_z=\Delta r_f$, so that
\begin{linenomath*}\begin{equation*}
    D_z\mathbf{z} \sim \bigo\left(\dfrac{1}{L_y L_z}\right) \, ,
\end{equation*}\end{linenomath*}
and all terms are of the same order of magnitude.
In general circulation models, these choices for $L_y$ and $L_z$ are generally
reasonable because anisotropy is encoded into the mesh.
It turns out that specifying the length scales as such provides for a
practical implementation, as the range parameter can be chosen as a simple
scaling to amplify or attenuate the correlation lengths.
One can then intuitively regard $\hat\rho$ as a nondimensional parameter that
controls the
``number of neighboring grid cells'' at which correlation decays to 0.1.
Our numerical experiments in section \ref{sec:matern_pig} make this clear.
