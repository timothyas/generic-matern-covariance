\section{A Nonstationary and Anisotropic Mat\'ern Correlation Operator}
\label{sec:matern_operator}

Here we propose to use the SPDE operator described by \citet{RSSB:RSSB777}
as a means to describe an anisotropic, nonstationary correlation model in a
similar manner to the diffusion-based methods described in
\cref{ssec:wc01_review}.
To do so, we employ the ``mapping method'' described by \citet{RSSB:RSSB777}
which we show for the general $M$th order SPDE in \cref{ssec:mapping_method}.
We then provide suggestions on how to parameterize the model for intuitive use
in practical applications in \cref{ssec:scaling_laplacian,ssec:range_parameter}.


\subsection{Mapping Method via Change of Variables}
\label{ssec:mapping_method}

In \citet{RSSB:RSSB777} it is suggested that solving the SPDE in a transformed
coordinate system can allow for a Mat\'ern class covariance
model that can easily incorporate anisotropy and nonstationarity.
To do so, consider solutions to the isotropic and stationary SPDE
(\cref{eq:spde_iso}), $\unih(\xh)$, to be defined in a transformed, or
``deformed'' \citep{sampson_nonparametric_1992}, space $\defdomain$.
Then, assume that we have a mapping $\defmap$ that maps between this transformed space
and our computational domain, $\domain$:
\begin{linenomath*}\begin{equation*}
    \defmap : \defdomain\ni\xh \rightarrow \x\in\domain \, .
\end{equation*}\end{linenomath*}
With this mapping, we can employ a change of variables
\citep{smith_change_1934} to rewrite the SPDE in the computational domain as:
\begin{linenomath*}\begin{equation*}
    \dfrac{1}{\defdet^M}
    \left(\deltah -
    \defdet\nabla\cdot
    \dfrac{\defjac(\x)\defjac(\x)^T}{\defdet}
    \nabla\right)^M\uni(\x) =
    \defdet^{-1/2}\W(\x) \, .
\end{equation*}\end{linenomath*}
Here we have defined the Jacobian as
\begin{linenomath*}\begin{equation*}
    \defjac(\x_0) \coloneqq
    \dfrac{\partial \defmap}{\partial \xh}\Big|_{\defmap^{-1}(\x_{0})} \, ,
\end{equation*}\end{linenomath*}
and for now we assume that $\defmap^{-1}(\x_0)$ is well defined.
For our purposes, this turns out to be the case, but this becomes clear when
$\defjac$ is defined in \cref{sec:matern_operator}.
%Notice that we have taken the exponent $M=1$, avoiding
%fractional or higher order operations for simplicity.
%All of the future formulations and experiments will make this assumption,
%although this can be relaxed in future work.
With the following definitions:
\begin{linenomath*}\begin{equation}
    K(\x) \coloneqq
    \dfrac{\defjac(\x)\defjac(\x)^T}{\defdet}
    \qquad
    \delta(\x) \coloneqq \dfrac{\deltah}{\defdet}
    \label{eq:matern_definitions}
\end{equation}\end{linenomath*}
the SPDE in \cref{eq:spde_iso} can be written in the computational domain's coordinate system as
\begin{linenomath*}\begin{equation}
    \Big(\delta(\x)- \nabla\cdot K(\x)\nabla\Big)^M \uni(\x) =
    \defdet^{-1/2}\W(\x) \, .
    \label{eq:spde_general}
\end{equation}\end{linenomath*}
We note as in \citet{RSSB:RSSB777} that this reproduces the deformation method
introduced in \citet{sampson_nonparametric_1992}.

Here, we propose to use this generic form to define a square root of the
correlation matrix in a similar fashion to
\citet{weaver_correlation_2001, mirouze_representation_2010,
carrier_background-error_2010}
as follows,
\begin{linenomath*}\begin{equation}
    \corrMat^{1/2} \coloneqq \normalizer
    \Big(\delta(\x)- \nabla\cdot K(\x)\nabla\Big)^{-M}
    \defdet^{-1/2} \, ,
    \label{eq:matern_operator}
\end{equation}\end{linenomath*}
where $\normalizer$ is
once again a variance-preserving normalization matrix defined by the operations
that precede it.
In this model,
anisotropy and nonstationarity are controlled by
$\defjac(\x)$ and $\deltah$, and in the following subsections we discuss how these can be assigned for
practical applications in geophysical inverse problems.
We note that in this discussion we loosely mix the use of
finite dimensional matrices and infinite dimensional operators
in order to ease the presentation, but we provide a more careful
derivation of their discretized forms in \cref{sec:discretization_matern}.

\subsection{Scaling the Laplacian Term}
\label{ssec:scaling_laplacian}

First, we address the Jacobian, $\defjac(\x)$, which appears in the tensor
$K(\x)$.
We illustrate how this operator controls correlation length scales and
motivate its parameterization with a simple scaling analysis.
Consider a 3D field $\uni(\x)\sim U$ that exhibits spatial variability at the
length scales, $L_x$, $L_y$, and $L_z$ in the direction of longitude, latitude,
and height, respectively,
where $L_x, L_y >> L_z$, such that the field exhibits highly
anisotropic fluctuations.
This is a common situation in large scale geophysical fluid
dynamics, where fields (e.g.\ temperature, velocity) exhibit length scales of
variability that are much greater in either horizontal dimension compared to the
vertical.
\footnote{
    We make a note on the terminology used here.
    In oceanography, the difference in horizontal and vertical scales
    is a result of the small aspect ratio, or the shallow fluid nature of the ocean.
    Because of the vast difference in scales, the horizontal dimensions are
    sometimes considered entirely independent of the vertical, and ``anisotropy''
    can be used to refer to heterogeneity between the horizontal components.
    However, here we use anisotropy to refer to the difference between horizontal
    and vertical scales, resulting from the small aspect ratio \citep{vallis2006}.
}

Without any rescaling, i.e.\ without $K$,
the Laplacian term in $\maternOp$ is imbalanced
\begin{linenomath*}\begin{equation}
    \begin{aligned}
        \nabla^2 \uni(\x)
            & \sim \dfrac{U}{L_x^2} + \dfrac{U}{L_y^2} + \dfrac{U}{L_z^2} \\
            & \simeq \dfrac{U}{L_z^2} \, .
    \end{aligned}
    \label{eq:iso_lap}
\end{equation}\end{linenomath*}
As a result, the correlation model will
have unrealistically large or small correlations in the horizontal or vertical.
Our goal is therefore to define the elements of $K$ such that each term is of
the same order of magnitude and
\begin{linenomath*}\begin{equation*}
    \nabla\cdot K\nabla \sim 3U \, .
\end{equation*}\end{linenomath*}

To achieve this balance between Laplacian terms, we suggest a straightforward,
perhaps obvious, specification of $\defjac$:
\begin{linenomath*}\begin{equation*}
    \defjac =
        \begin{pmatrix}
            L_x & 0 & 0     \\
            0 & L_y & 0     \\
            0 & 0   & L_z   \\
        \end{pmatrix} \, ,
\end{equation*}\end{linenomath*}
where we simply ignore the off-diagonal elements of $\defjac$.
The determinant in this case is $\defdetnox = L_xL_yL_z$ and
according to the definitions in \cref{eq:matern_definitions}:
\begin{linenomath*}\begin{equation*}
    K =
        \begin{pmatrix}
            1/L_z & 0 & 0     \\
            0 & 1/L_z & 0     \\
            0 & 0   & L_z/(L_xL_y)   \\
        \end{pmatrix} \, ,
\end{equation*}\end{linenomath*}
so that
\begin{linenomath*}\begin{equation*}
    \nabla\cdot K(\x)\nabla\uni(\x) \sim \dfrac{3}{L_xL_yL_z}U \, .
\end{equation*}\end{linenomath*}
The key is that $K$ scales each term in the Laplacian so that they are
approximately the same order of magnitude, and the operator is balanced in either direction.
%This example can further be extended by allowing the length scales $L_1$ and
%$L_2$ to vary in space, such that correlations are nonstationary.

At this point, we must prescribe values for $L_x(\x)$, $L_y(\x)$, and $L_z(\x)$ to fill
$\defjac(\x)$.
Considering the discretization of the Laplacian, a simple choice for these can
be based on the underlying grid of the general circulation model.
For the numerical experiments in this paper, we choose
$L_x(i,j) = \Delta x_g(i,j)$, $L_y(i,j) = \Delta y_g(i,j)$,
and $L_z(k) = \Delta r_f(k)$ (\cref{fig:mitgcm_grid}) where we have switched
from the spatial coordinate $\x\in\domain$ to the computational grid indices $i$, $j$, $k$.
We consider the choice to use the grid elements directly to be reasonable
because anisotropy and nonstationarity is usually encoded into the grid.
For instance, regarding nonstationarity, it is reasonable to assume that
correlation length scales near the surface of the ocean are relatively short as
a result of the locality of atmosphere-ocean interactions that occur there.
On the other hand, near the ocean floor motion is presumed to be more
quiescient, and correlation length scales are correspondingly longer.
Many ocean models
\citep[e.g.][]{nguyen_arctic_2021, forgetECCOv4}
incorporate these assumptions into the vertical grid resolution.


\subsection{The Range Parameter}
\label{ssec:range_parameter}

The other term in \cref{eq:spde_general} to be defined is $\deltah$.
Here we resort to the empirical relation suggested in
\citet{RSSB:RSSB777}:
\begin{linenomath*}\begin{equation*}
    \rangeh = \sqrt{\dfrac{8\meandiff}{\deltah}} \, .
\end{equation*}\end{linenomath*}
The so-called range parameter, $\rangeh$, defines the
distance between two points at which correlation drops to 0.14.
We note that the range parameter used here is larger by a factor of two than
what is commonly often used, for instance this is twice the scale parameter used
by \citet{mirouze_representation_2010}.
We prefer the definition used here because the relation described above provides an intuitive
understanding of $\rangeh$.

%While it may seem like we have swapped one unknown for another, it is usually easier to
%define correlation length scales than simply guessing values for $\deltah$.
With this relation and our choice for $\defjac(\x)$ in
\cref{ssec:scaling_laplacian} we have
\begin{linenomath*}\begin{equation*}
    \begin{aligned}
        \delta(i,j,k) &= \dfrac{8\meandiff}{\rangeh^2\defdetd} \\
                      &= \dfrac{8\meandiff}{\rangeh^2 L_x(i,j,k)L_y(i,j,k) L_z(i,j,k)} \, .
    \end{aligned}
\end{equation*}\end{linenomath*}
With this choice, both this term and the Laplacian term are now of a similar order of
magnitude
\begin{linenomath*}\begin{equation*}
    D_{\delta}\unis \sim \nabla \cdot K \nabla \unis \sim
    \bigo\left(\dfrac{\meandiff}{L_xL_yL_z}\right) \, ,
\end{equation*}\end{linenomath*}
where we note that $\meandiff \sim \bigo(1)$ is a reasonable assumption since
$\meandiff=5/2 \implies M=2$ gives a good approximation to a Gaussian
correlation when $d=3$, see \red{FIG}.

With $L_x$, $L_y$, and $L_z$ defined by the model grid elements, the elliptic
operator in \cref{eq:matern_operator} is effectively nondimensionalized.
We assert that with this definition, one can use this operator on an
anisotropic and spatially varying computational grid armed with the intuition that
$\rangeh$ is a nondimensional parameter that controls the
``number of neighboring grid cells'' at which correlation decays to 0.14.
Our numerical experiments in \cref{sec:llc90} show that this is a good
approximation in the case of a realistic global ocean model grid.

