\documentclass[alpha-refs]{qjrms/wiley-article}

\usepackage{url}
\usepackage{lineno}
\usepackage{soul}

\usepackage{graphicx}
\usepackage[space]{grffile}
\usepackage{latexsym}
\usepackage{textcomp}
\usepackage{longtable}
\usepackage{tabulary}
\usepackage{booktabs,array,multirow}
\usepackage{amsfonts,amsmath,amssymb}
\usepackage{natbib}
\usepackage{etoolbox}
\makeatletter
% This gives an error ...
%\patchcmd\@combinedblfloats{\box\@outputbox}{\unvbox\@outputbox}{}{%
%  \errmessage{\noexpand\@combinedblfloats could not be patched}%
%}%
\makeatother
% You can conditionalize code for latexml or normal latex using this.
\newif\iflatexml\latexmlfalse
\providecommand{\tightlist}{\setlength{\itemsep}{0pt}\setlength{\parskip}{0pt}}%

\AtBeginDocument{\DeclareGraphicsExtensions{.pdf,.PDF,.eps,.EPS,.png,.PNG,.tif,.TIF,.jpg,.JPG,.jpeg,.JPEG}}

\usepackage[utf8]{inputenc}
\usepackage[english]{babel}

% --- Custom additions
\usepackage{mathtools} % for \coloneqq
\usepackage{color}
\usepackage{bm}
\usepackage{adjustbox}
\usepackage[percent]{overpic}


% Definitions
\usepackage{ide}

\usepackage{hyperref}
\hypersetup{colorlinks=false,pdfborder={0 0 0}}
\usepackage[capitalise,noabbrev]{cleveref}

\newcommand{\red}[1]{\textcolor{red}{#1}}
\newcommand{\blue}[1]{\textcolor{blue}{#1}}

\newcommand{\ca}{\ensuremath{\cos(\alpha)}}
\newcommand{\cb}{\ensuremath{\cos(\beta)}}
\newcommand{\cat}{\ensuremath{\cos^2(\alpha)}}
\newcommand{\cbt}{\ensuremath{\cos^2(\beta)}}
\newcommand{\sia}{\ensuremath{\sin(\alpha)}}
\newcommand{\sib}{\ensuremath{\sin(\beta)}}
\newcommand{\siat}{\ensuremath{\sin^2(\alpha)}}
\newcommand{\sibt}{\ensuremath{\sin^2(\beta)}}
\newcommand{\ta}{\ensuremath{\tan(\alpha)}}
\newcommand{\tb}{\ensuremath{\tan(\beta)}}
\newcommand{\tbt}{\ensuremath{\tan^2(\beta)}}


% Define some citation abbreviations
\defcitealias{RSSB:RSSB777}{L11}

% From qjrms
\renewcommand{\sfdefault}{ptm}
\renewcommand{\familydefault}{\sfdefault}

\papertype{Supplemental Material}
\renewcommand{\thefigure}{S\arabic{figure}}

% --- Header
\title{A Rotated Mapping for the Mat\'ern Class Correlation Operator
}
\author{Timothy A. Smith}

\runningauthor{Smith}

\begin{document}

\maketitle

Here we show the derivation for a rotated tensor that could be used in a
Mat\'ern type correlation model.
This is motivated by the example shown in \citet{weaver_correlation_2001} for
the explicit diffusion operator.
However, we were unable to get this implementation to be stable and had to
abandon the effort due to time constraints.

We start by applying the rotation tensor $\mathbf{S}$ to the Jacobian $\defjac$
\begin{linenomath*}\begin{equation*}
    \tilde{\defjac} \coloneqq \mathbf{S} \defjac \, ,
\end{equation*}\end{linenomath*}
where recall that we're using simply
\begin{linenomath*}\begin{equation*}
    \defjac =
        \begin{pmatrix}
            L_x & 0 & 0 \\
            0 & L_y & 0 \\
            0 & 0 & L_z \\
    \end{pmatrix} \, .
\end{equation*}\end{linenomath*}
We suggest to use the rotation tensor defined by \citep{redi_oceanic_1982}
\begin{linenomath*}\begin{equation*}
    \mathbf{S} \coloneqq
        \begin{pmatrix}
            \ca \cb & - \sia & - \sib\ca \\
            \sia \cb & \ca & - \sib\sia \\
            \sib & 0 & \cb \\
        \end{pmatrix}
\end{equation*}\end{linenomath*}
where the angles $\alpha$ and $\beta$ describe rotations between the geodesic
and isopycnal coordinate system in the $z$ and $y$ directions.
A more useful interpretation is
\begin{linenomath*}\begin{equation*}
    \begin{aligned}
        S_x &= \ca\tb = -\sigma_x / \sigma_z \\
        S_y &= \sia\tb = -\sigma_y / \sigma_z
    \end{aligned}
\end{equation*}\end{linenomath*}
where $\sigma$ is potential density and e.g.\ $\sigma_x$ is it's derivative in
the $x$ direction.

With this rotation,
\begin{linenomath*}\begin{equation*}
    \tilde{\defjac} =
        \begin{pmatrix}
            L_x \ca\cb & -L_y\sia & -L_z \sib \ca \\
            L_x \sia\cb & L_y \ca & -L_z \sib \sia \\
            L_x \sib & 0 & L_z \cb \\
        \end{pmatrix}
\end{equation*}\end{linenomath*}
and one can confirm that $\text{det}\left(\tilde{\defjac}\right) =
\text{det}\left(\defjac\right) = L_x L_y L_z$, as should be the case given that
$\mathbf{S}$ is a unitary rotation.
\begin{linenomath*}\begin{equation*}
    \tilde{\defjac}\tilde{\defjac}^T =
        \begin{pmatrix}
            \kappa_{11} & \kappa_{12} & \kappa_{13} \\
            \kappa_{21} & \kappa_{22} & \kappa_{23} \\
            \kappa_{31} & \kappa_{32} & \kappa_{33} \\


        \end{pmatrix}
\end{equation*}\end{linenomath*}
where
\begin{linenomath*}\begin{equation*}
    \begin{aligned}
        \kappa_{11} &=
            \left[L_x^2\cat\cbt + L_y^2\siat + L_z^2\cat \sibt \right] \\
        \kappa_{12} &=
            \left[L_x^2\ca\sia\cbt - L_y^2\ca\sia + L_z^2\ca\sia\sibt \right] \\
        \kappa_{13} &=
            \left[L_x^2\ca\cb\sib - L_z^2\ca\cb\sib \right] \\
%
        \kappa_{21} &=
            \left[L_x^2\ca\sia\cbt - L_y^2 \ca\sia + L_z^2 \ca\sia\sibt \right] \\
        \kappa_{22} &=
            \left[L_x^2\siat\cbt + L_y^2\cat + L_z^2\siat \sibt\right] \\
        \kappa_{23} &=
            \left[L_x^2\sia\cb\sib - L_z^2\sia\cb\sib \right] \\
%
        \kappa_{31} &=
            \left[L_x^2\ca\cb\sib - L_z^2\ca\cb\sib \right] \\
        \kappa_{32} &=
            \left[L_x^2\sia\cb\sib - L_z^2\sia\cb\sib \right] \\
        \kappa_{33} &=
            \left[L_x^2\sibt + L_z^2\cbt\right] \\
    \end{aligned}
\end{equation*}\end{linenomath*}
To simplify the situation, take $L_x = L_y$ and note the super useful identity
$1/\cbt = 1 + \tbt$.
Then,
\begin{linenomath*}\begin{equation*}
    \begin{aligned}
        \tilde{K}
        &=
        \dfrac{\tilde{\defjac}\tilde{\defjac}^T}{\text{det}\left(\tilde{\defjac}\right)}
        \\
        &= \dfrac{1}{L_z\left(1 + S_x^2+s_y^2\right)}
            \begin{pmatrix}
                1 + \gamma^2 S_x^2 + S_y^2 &
                -S_x S_y(1-\gamma^2) &
                S_x(1-\gamma^2) \\
%
                -S_xS_y(1-\gamma^2) &
                1+S_x^2 + \gamma^2S_y^2 &
                S_y(1-\gamma^2) \\
%
                S_x(1-\gamma^2) &
                S_y(1-\gamma^2) &
                S_x^2 + S_y^2 + \gamma^2 \\
            \end{pmatrix}
    \end{aligned}
\end{equation*}\end{linenomath*}
where $\gamma \coloneqq L_z/L_x$.
We can then make the small angle approximation, where
$S_x^2$, $S_y^2$, $S_xS_y << 1$, and owing to the aspect ratio, $\gamma^2<<1$.
Then
\begin{linenomath*}\begin{equation*}
    \tilde{K} = \dfrac{1}{L_z}
        \begin{pmatrix}
            1 & 0 & S_x \\
            0 & 1 & S_y \\
            S_x & S_y & S_x^2+S_y^2+\gamma^2 \\
        \end{pmatrix} \, .
\end{equation*}\end{linenomath*}
This is what is implemented.
The remaining technical detail, and most probable location of the bug, is in
the implementation of the off-diagonal Laplacian elements.
For these to be correct, the derivatives need to be interpolated to different
interfacial points than they are at (e.g.\ the x-derivatives need to be
interpolated to the upper/lower z-interfacial positions).
I derived these in a rush, and it could be redone.

\bibliography{references}
\end{document}
