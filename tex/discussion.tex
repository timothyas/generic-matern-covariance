\section{Discussion}
\label{sec:matern_discussion}

In this work we have shown a general methodology for applying a Mat\'ern type
correlation function within a domain with complex boundaries, while achieving
nonstationary and anisotropic correlation length scales.
To summarize, the general procedure is as follows.
First, one chooses a \red{normalization length scale} for each dimension, thereby
defining (the Jacobian of) a mapping between a space where correlation is
isotropic and stationary, and the more complex domain with nonstationary and
anisotropic correlation length scales.
Next, one must choose a range parameter, determining the distance relative to
the \red{normalizing length scales} at which correlation drops to 0.14.
Finally, one selects the shape of the correlation structure, which also sets the
number of times the elliptic PDE must be solved.

Our presentation has focused on the practical application of this
correlation operator within an ocean general circulation model.
As such, we set the \red{normalizing length scales} based on the local grid
scale.
With this setting, the range parameter is shown to be a highly intuitive dial,
controlling correlation length scales as a simple function of the number of
neighboring grid cells.
Using this definition was further shown to be beneficial at the equator, for
example, because grid scale refinements there result in relatively
longer zonal than meridional correlation length scales and this is \red{observed
to be physically realistic}.
However, we recognize that there could be features that are desirable to capture
in a correlation model that are not represented in the definition of the
underlying model grid.
In this case, the \red{normalizing length scales} could be further tuned with
local factors or functions to achieve these desired features.
Alternatively, the \red{length scales} could be set entirely independently of
the grid, for instance as a function of the local Rossby radius of deformation.

A key feature of the correlation model shown here is the $\rangeh$ and $M$
control the correlation length scale and shape \textit{separately}.
We consider this to be an attractive feature when compared to the implicit
diffusion approach \red{albeit, without changing the order of the laplacian or
any parameters as described in} \citet{weaver_diffusion_2013}.
In this model, at a given ``length scale'' the distance at which correlation
drops to a specified threshold still changes with $M$
(\cref{fig:correlation_comparison}, and also
\citep[Figs. 1 and 2 of][]{guillet_modelling_2019}).
We note that in the Mat\'ern correlation model presented here that the parameter
$\deltah$ changes with $M$ while in the implicit diffusion approach, $\deltah
\rightarrow 1$.
Apparently the simple variation in this parameter is enough to balance the
multiple applications of $\maternop$, such that the resulting correlation
structure maintains a consistently identifiable length scale via $\rangeh$.

As noted in \citet{mirouze_representation_2010,carrier_background-error_2010}, a
drawback to the explicit diffusion approach from
\citet{weaver_correlation_2001} is that it requires many iterations to satisfy
numerical stability.
We note specifically in our experimentation with this model on the LLC grid
as implemented in the MITgcm \citep{campin_mitgcmmitgcm_2021}, we have found
the number of iterations required for numerical stability to be roughly a factor
of three larger than the necessary (but insufficient) bound for numerical
stability.
We therefore find approaches based on the implicit solution of a PDE to be more
straightforward, as it is more intuitive to specify a solution tolerance
rather than guess the number of iterations required for convergence.
Moreover, our numerical experiments indicate that the solution can be imprecise
(to a tolerance of $\sim10^{-3}$) and therefore highly efficient.
Finally, because the correlation model shown here is formulated through an
inverse elliptic operator, we have access to the inverse correlation operator,
which could be used directly as regularization while solving an inverse problem
\citep[e.g.][]{bui-thanh_computational_2013}, or for the specification of
spatially correlated observations as in \citet{guillet_modelling_2019}.

For some applications it could be desirable to specify oscillating or ``lobed''
correlation models, which can be achieved with the explicit or implicit
diffusion models \citep{weaver_correlation_2001,weaver_diffusion_2013}.
We suggest that such extensions are possible for the Mat\'ern type correlation
operator shown here, based on results shown by \citet{RSSB:RSSB777} in the
complex plane with a tunable oscillation parameter.
These more general shapes could be explored in future work for the case of
multi-dimensional fields as shown here.
