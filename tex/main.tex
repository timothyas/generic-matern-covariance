\documentclass[draft]{Feb-3-22-latex-templates/agujournal2019}

% --- Journal Provided
\usepackage{url}
\usepackage{lineno}
\usepackage{soul}
\journalname{Journal of Advances in Modeling Earth Systems (JAMES)}
\draftfalse
%\linenumbers

% --- Custom additions
\usepackage{amsmath, amssymb}
\usepackage{mathtools} % for \coloneqq
\usepackage{color}
\usepackage{graphicx}
\usepackage{natbib}
\usepackage{bm}

% Definitions
\usepackage{oden}

\usepackage{hyperref}
\hypersetup{
    colorlinks=true,
    linkcolor=black,
    filecolor=black,
    urlcolor=black,
    citecolor=black
}
\usepackage[capitalise]{cleveref}

\newcommand{\red}[1]{\textcolor{red}{#1}}
\newcommand{\blue}[1]{\textcolor{blue}{#1}}

% Removes redundant "appendix"
\crefname{appendix}{}{}

% Define some citation abbreviations
\defcitealias{RSSB:RSSB777}{L11}

\begin{document}

% --- Header
\title{A Practical Gaussian Covariance Formulation for Applications with
       Anisotropic and Nonstationary Fields}
\authors{Timothy A. Smith\affil{1,2,3}, Patrick Heimbach\affil{3,4,5}}

\affiliation{1}{CIRES}
\affiliation{2}{PSL}
\affiliation{3}{Oden}
\affiliation{4}{UTIG}
\affiliation{5}{JSG}

% --- Key points, abstract, plain language summary
\begin{keypoints}
    \item We derive a generic covariance model appropriate for anisotropic,
        nonstationary fields \red{This is the title...}
    \item The formulation presents a number of computational advantages compared
        to previous work, such as easy application of to both the covariance or
        precision operators (its inverse)
    \item Numerical experiments agree well with theory... \red{more on this...}
\end{keypoints}

\begin{abstract}
    A key component of data assimilation methods is the specification of
    univariate spatial correlations, which appear in the background-error
    covariance.
    For realistic problems in meteorology and oceanography,
    correlation length scales are nonstationary (variable in space) and
    anisotropic (variable in each direction).
    Variational approaches typically use an operator to enforce correlation
    length scales, and thus the operator must be designed to capture
    desired levels of nonstationarity and anisotropy.
    For systems with complex boundaries, such as the ocean, it is natural
    to use a filtering approach based on the application of an elliptic,
    Laplacian-like operators.
    Here we show how an elliptic operator can be formulated to capture a general
    Mat\'ern-type correlation structure for practical use in variational data
    assimilation.
    We show how nonstationarity and anisotropy can be encoded into the operator
    via a simple change of variables based on user-defined
    normalization length scales.
    The change of variables defines a mapping between the computational domain
    and a space where the isotropic and stationary Mat\'ern correlation function
    applies.
    In addition to the mapping, two other hyperparameters \textit{separately}
    control the correlation length scale (i.e.\ range) and shape.
    As a practical use-case, we apply the correlation operator to a global ocean model.
    We show that when the normalizing length scales are chosen
    based on the local grid scale, the range parameter has an intuitive
    interpretation as the number of neighboring grid cells at which correlation drops to 0.14.
    Finally, the correlation model is additionally shown to be computationally efficient
    in two aspects.
    First,
    the required linear solve can be performed with a high tolerance  ($\sim$~$10^{-3}$)
    while still achieving the correct statistics, requiring relatively few
    iterations to converge.
    Secondly, the shape of the correlation function is determined by the number
    of times the operator is applied, but, owing to the generic Mat\'ern form,
    this also improves the diagonal dominance of the matrix form.
    Thus, it is shown that the model can actually be \textit{more} efficient with more
    applications of the inverse elliptic operator.
\end{abstract}

\section*{Plain Language Summary}
Mat\'ern, Weaver, and Courtier walk into a bar...

\section{Introduction}
\label{sec:intro}

High dimensional geophysical inverse problems,
such as numerical weather prediction and
oceanographic state estimation, are typically ill-posed due to the
sparsity of data relative to the size of the control vector.
A classical method for handling this ill-posedness is to prescribe
some type of regularization in order to ``spread'' information to the uninformed
regions and variables in the control vector \citep[e.g.,][]{wunsch_discrete_2006}.
In the Bayesian interpretation of the inverse problem, this regularization
is defined so that it represents the prior uncertainty or background-state error
\citep[e.g.,][]{bui-thanh_computational_2013}.
Ideally, this uncertainty captures the true error in the background state, but
for all practical applications the background error must be
approximated.
Moreover, to make the problem computationally tractable, it is often assumed
that the background error is governed by Gaussian statistics, such that the
uncertainty is fully described by a covariance matrix.

For realistic data assimilation (DA) problems in meteorology and
oceanography, a well formed background error will contain covariance relationships between
different variables (e.g., between temperature and velocity components),
it will have spatially dependent length scales of covariation
(i.e.\ nonstationarity or inhomogeneity),
and it will respect the system's anisotropy, such that length scales of covariance differ
appropriately in each direction \citep[e.g.,][]{bannister_review_2008-1}.
Within variational DA systems, the background error covariance is
usually represented as an operator so that it can be applied efficiently during
an iterative optimization algorithm.
Thus, for an operator-based covariance model to be useful in this context, it must be
able to respect multivariate, nonstationary, and anisotropic features that are
necessary for the given problem setting.

Typically, the background state error covariance is decomposed into two
operators following \citet{derber_reformulation_1999}.
The first, ``balance'' operator captures multivariate (i.e.\ cross-variable)
covariance information while the second captures ``unbalanced'' (i.e.\
univariate) covariance information.
Our focus is on the latter operator, but we note the review by
\citet{bannister_review_2008-2} which outlines balance operators used in
atmospheric DA, and e.g., \citet{weaver_multivariate_2005,moore_regional_2011-1} for oceanographic
examples.

Univariate covariance operators are further decomposed into two stages, where spatial
correlations are specified first, and then scaled to the appropriate amplitude.
In atmospheric DA, it is common to transform the control vector
into a wavelet or spherical harmonic basis in order to specify the background
error \citep[e.g.,][]{bannister_review_2008-2}.
However, for systems with complex boundaries like the ocean, it is far more
straightforward to formulate the correlation operator in terms of the original,
physical domain.
Considering methods that operate in the physical or grid space, there are
generally two classes of correlation models that are commonly used.
The first class is encapsulated by explicit functional forms, including, for
example, the functions developed by
\citet{gaspari_construction_1999,gneiting_correlation_1999,gaspari_construction_2006},
which have the benefit of providing compact support.
The second class of correlation models can generally be described as a filtering
approach.
\citet{purser_numerical_2003-2,purser_numerical_2003-1}
show correlation functions based on recursive filters, and
\citet{dobricic_oceanographic_2008} extend this to be used with complex boundaries.
More recently, \citet{purser_multigrid_2022} show a ``beta
filter'' approach which enables compact support and a highly generalizable
correlation shape via an efficient multigrid approach.
Alternatively, within this class of models are those based on the solution of a
differential equation.

Correlation models that are based on the solution to differential operators have
several advantages.
Most importantly, these operators handle complex boundaries naturally and the
infrastructure required to obtain their solution typically exist within
the underlying numerical model.
Within oceanographic state estimation, a widely used framework is based on the
solution to the diffusion equation
\citep[e.g.,][]{nguyen_arctic_2021,forgetECCOv4,blockley_recent_2014,moore_regional_2011-1,daget_ensemble_2009,muccino_inverse_2008,di_lorenzo_weak_2007,weaver_three-_2003}.
The diffusion-based framework relies on either an explicit, pseudo-time stepping
method \citep{weaver_correlation_2001} or an implicit solution
\citep{mirouze_representation_2010,carrier_background-error_2010,weaver_diffusion_2013},
where the correlation structure underlying the solution corresponds to either a
Gaussian or more general auto-regressive function, respectively.
Alternatively, \citet{RSSB:RSSB777} show an explicit link between the numerical solution
to a stochastic elliptic partial differential equation and a Mat\'ern-type covariance
model.
As of yet, however, it has remained unclear how this model could be used to
specify univariate correlations with appropriate nonstationarity and anisotropy
for an operational variational DA system.

Here, we extend the work by \citet{RSSB:RSSB777}
to show how the framework can be used within variational DA.
Our emphasis is on oceanographic applications, although the methodology is more
general.
We show how the mapping method introduced by \citet{RSSB:RSSB777}
can be used to formulate a correlation operator that respects anisotropy and
nonstationarity in a way that is relevant to many ocean general circulation
models.
At the core of this model there are two parameters, which \textit{separately}
control the shape and length scale of the correlation model.
We show that the parameter controlling the length scale can be interpreted
intuitively as the ``number of neighboring grid cells'' at which correlation
drops to an expected value.
Our numerical experiments show that correlations derived from the isotropic,
stationary Mat\'ern correlation function can be mapped to a generic ocean model
grid in terms of this range parameter.
It is therefore straightforward to use this approach to achieve desirable
anisotropic and nonstationary statistics by simply tuning these two
parameters.

% Is this really necessary?
The paper is laid out as follows.
In \cref{sec:review} we give some context for how univariate correlation models
are used in variational data assimilation.
We provide a review of diffusion based correlation operators, and
then review the Mat\'ern type covariance developments in \citet{RSSB:RSSB777}
which we build on.
In \cref{sec:matern_operator} we show how the Mat\'ern model can be mapped from
its isotropic, stationary form into a more complex computational domain.
We then provide suggestions for parameterizing the model so that it can
intuitively capture anisotropy and nonstationarity.
In \cref{sec:llc90} we show numerical results of this correlation model
applied to the global ocean, using the ``Lat-Lon-Cap'' grid introduced by
\citet{forgetECCOv4}.
Finally, in \cref{sec:matern_discussion} we provide some discussion on the
advantages of this model, and a general comparison to the widely used
diffusion based models.


\section{Mathematical Context and Background}
\label{sec:review}

In order to provide some mathematical context for our covariance model
developments, we first outline the generic inverse problem that is central to
a variety of applications such as numerical weather prediction or state
estimation.
Our notation closely follows \citet{ide_unified_1997}, and we note that matrix
notation is used to describe the problem, but these matrices are never formed
explicitly.
Rather, all matrices presented can be described as operators that can be applied
scalably in high dimensional inverse problems.
We then review how the WC01 correlation model fits into the background state
error covariance.
Finally, we discuss the developments from \citet{RSSB:RSSB777}, outlining the
connection between the solution to a Stochastic PDE (SPDE) and a Gaussian random
field with Mat\'ern type covariance.
This review sets up our development of an anistropic, nonstationary covariance
model which is presented in Section \ref{sec:matern_operator}.


\subsection{Inverse Problem Formulation}
\label{ssec:da_formulation}

We consider the general problem of finding the optimal control vector,
$\params$, which minimizes the regularized model-data misfit cost function
\begin{linenomath*}\begin{equation*}
    \cf(\params) =
        \dfrac{1}{2}||\pto(\params) - \data||_{\obsCovMat^{-1}}^2
        +
        \dfrac{1}{2}||\params - \priorParams||_{\priorCovMat^{-1}}^2 \, .
\end{equation*}\end{linenomath*}
The solution to this inverse problem, $\paramsMAP$, arises from a tradeoff between fitting the
observational data, $\data$, as governed by the parameter to observable map
$\pto(\cdot)$, and minimizing deviations from the background-state $\priorParams$.
This tradeoff is governed by the two error covariances, $\obsCovMat$ and
$\priorCovMat$, which dictate how much deviation is acceptable in either term.
On the one hand, the observational error covariance matrix
$\obsCovMat$ represents our uncertainty
in the observational data, together with our confidence in the model's ability
to represent the observed values.
On the other hand, the background-state covariance matrix $\priorCovMat$
represents our uncertainty in the prior estimate or background state,
$\priorParams$.

In this work, our focus is on the development of the background-state error
covariance, $\priorCovMat$.
In the general case, the control vector $\params$ could be multivariate,
including initial conditions of the system state, uncertain boundary
conditions, or uncertain parameter fields.
Here we employ the decomposition proposed by
\citet{derber_reformulation_1999}
in order to separate the multivariate (i.e.\ cross-variable)
covariance relationships from the univariate (i.e.\ assumed
independent) covariance relationships.
Specifically, the background-state covariance is decomposed as follows
\begin{linenomath*}\begin{equation*}
    \priorCovMat \coloneqq \balanceOperator\unbalancedPriorCovMat\balanceOperator^T \,
    ,
\end{equation*}\end{linenomath*}
where $\balanceOperator$ is a balance operator that deals with the
cross-variable correlations.
The matrix $\unbalancedPriorCovMat$ describes the covariance for the unbalanced
variables and has a block-diagonal structure, such that each
covariance is described independently.
The unbalanced covariance is further factored as
\begin{linenomath*}\begin{equation*}
    \unbalancedPriorCovMat \coloneqq \Sigma \corrMat \Sigma
\end{equation*}\end{linenomath*}
where $\Sigma$ is a diagonal scaling matrix, containing the desired pointwise
standard deviation values and $\corrMat$ is a block diagonal correlation matrix,
describing each variable's independent correlation structure.
To be concrete, this can be viewed as
\begin{linenomath*}\begin{equation*}
    \unbalancedPriorCovMat =
    \begin{pmatrix}
        \Sigma_\alpha\corrMat_\alpha\Sigma_\alpha & 0 & \cdots & 0 \\
        0 & \Sigma_\beta\corrMat_\beta\Sigma_\beta & \cdots & 0 \\
        0 & 0 & \ddots & 0  \\
        0 & 0 & \cdots & \Sigma_\gamma\corrMat_\gamma\Sigma_\gamma \\
    \end{pmatrix}
\end{equation*}\end{linenomath*}
where $\alpha, \beta, \gamma$ are placeholders for unique variables, and each
$\Sigma_\alpha, \corrMat_\alpha$ pair describes the amplitude of pointwise standard deviation
and spatial correlation for each variable.


\subsection{The WC01 Correlation Model}
\begin{itemize}
    \item Talk about WC01 operator
    \item Discuss why it works
    \item Discuss some disadvantages
\end{itemize}


\subsection{Old intro}
A fundamental requirement for the inverse problem that we address in this work
is the specification of a prior distribution, $\priorDist(\params)$.
Recall from chapter XX that the multivariate control vector
$\params$ consists of the
potential temperature, salinity, and zonal velocity at the western boundary of
the domain:
$\params \coloneqq [\thetaParams^T,\saltParams^T,\uvelParams^T]^T \in\paramSpace$.
As is common in large scale geophysical inverse problems, we specify the prior
distribution for the control vector
to be Gaussian: $\priorDist(\params) \coloneqq \mathcal{N}(\params_0, \priorCovMat)$.

Our first objective in the inverse problem is to specify the prior
covariance $\priorCovMat$.
To do this, we obtain univariate prior covariances for
each individual field: $\thetaPriorCovMat$, $\saltPriorCovMat$,
$\uvelPriorCovMat$, which are then stacked block-diagonally to form
$\priorCovMat$ (see below for details).
Our primary focus in this chapter is to describe the generic formulation for
each of these univariate covariance matrices.
To facilitate the discussion, we refer to a generic univariate control variable
$\uni(\x)$ (or $\unis$ upon discretization), which in our specific case is a
placeholder for temperature, salinity, and zonal velocity, and is similarly
applicable to other inverse problems.
We use the general methodology outlined here to specify a prior covariance in
chapter XX.

In oceanographic inverse problems, covariance models must address at least
these three issues.
\begin{enumerate}
    \item Irregular boundaries imposed by continents
    \item Anisotropy due to the shallow fluid-like nature of the ocean
    \item Multivariate control parameters
\end{enumerate}
It is common to use covariance models based on differential operators in order
to handle irregular boundaries, and the question is then how to address the
other two issues within a differential equation.

A common approach to specifying the prior covariance in oceanographic inverse
problems is based on a generalized
diffusion equation \citep{weaver_correlation_2001}.
In this chapter, we outline an alternative approach that brings some practical
advantages which are discussed in section \ref{sec:matern_discussion}.
%#that has a similar structure to the
%#this model, but employs a version of the differential
%#operator presented in \citet{RSSB:RSSB777}.
%The general methodology is as follows.
We define a differential operator that can be represented by the matrix $C$,
that follows the flexible development from \citet{RSSB:RSSB777}.
We note that the matrix form is used for convenience, but that matrices are
never explicitly formed.
We show that the operator $C$ specifies the covariance matrix $CC^T$, that
is almost identical to a Mat\'ern covariance aside from boundary affects
imposed by $C$.
We then augment this differential operator with a sequence of factors that
are suggested by \citet{weaver_correlation_2001}.
That is, we incorporate the sequence of operations:
$\Sigma X C $, where
\begin{linenomath*}\begin{equation*}
    X \coloneqq \text{diag}\left\{ 1/\hat{\sigma}_{i}\right\}_{i=1}^{N}
\end{equation*}\end{linenomath*}
is a
normalization matrix computed from the pointwise marginal variance at grid cell
$i$: $\hat{\sigma}^2_{i}$, and
\begin{linenomath*}\begin{equation*}
    \Sigma \coloneqq \text{diag}\left\{\sigma_\uni\right\}_{i=1}^{\nuni}
\end{equation*}\end{linenomath*}
is the specified magnitude of prior uncertainty (standard deviation) for a
generic univariate parameter field $\unis\in\uniSpace$ (e.g.\ initial temperature).
With these definitions, $XCC^TX$ is a correlation matrix, and
\begin{linenomath}\begin{equation}
    \Gamma_\uni \coloneqq \Sigma X C C^T X \Sigma =
    \Gamma_\uni^{1/2}\Gamma_\uni^{T/2}
\end{equation}\end{linenomath}
defines the covariance for a generic univariate parameter field $\unis$.

The full prior covariance for a general multivariate application is then formed
by specifying individual covariance operators as above, and stacking them
together block diagonally.
To be concrete, the prior covariance for the specific inverse problem in this
work, i.e. for the steady state temperature, salinity,
and velocity fields at the open boundary of the computational domain,
is formulated as
\begin{linenomath*}\begin{equation*}
    \priorCovMat \coloneqq
    \begin{pmatrix}
        \thetaPriorCovMat & & \\
        & \saltPriorCovMat & \\
        & & \uvelPriorCovMat \\
    \end{pmatrix} \, .
\end{equation*}\end{linenomath*}
To keep this chapter general, we focus on specifying the covariance for a generic
univariate control parameter, $\uni(\x)$ which can be considered a placeholder
for each variable temperature, salinity, and velocity separately.
We wait until
chapter xx to discuss the specification of the prior
for each parameter, as this is becomes specific to our application.
Finally, we note that nonzero off diagonal terms in $\priorCovMat$ above (or an
additional operator) could be used to specify cross covariance between each
variable.
We consider this future work, and discuss potential options in
section XX.

In the following sections we review the general Mat\'ern type covariance
form that is suggested by \citet{RSSB:RSSB777}.
We then develop the differential operator, $C$, that
forms the backbone of our covariance model.
We show that our numerical implementation of the covariance model produces correlation
length scales that are expected from the analysis in section
\ref{sec:matern_operator}.
We conclude by discussing some advantages that this approach offers.


\section{Review of the Mat\'ern class covariance}
\label{sec:matern_review}

In this section we review the link between an elliptic stochastic partial
differential equation (SPDE) and Gaussian random fields.
The Mat\'ern covariance function between two points, $\xh_1,\xh_2\in\defdomain =
\ndspace$ can be expressed as:
\begin{linenomath}\begin{equation}
    c(\xh_1,\xh_2) = \dfrac{\sigma^2}{2^{\meandiff-1}
    \mathcal{G}(\meandiff)}
    \Big(\sqrt{\deltah} ||\xh_2-\xh_1||\Big)^\meandiff
    \mathcal{B}_\meandiff
    \Big(\sqrt{\deltah} ||\xh_2-\xh_1||\Big) \, .
    \label{eq:matern_covariance_iso}
\end{equation}\end{linenomath}
Here $||\cdot||$ as the Euclidean norm in $\defdomain$,
$\mathcal{G}$ is the Gamma function,
$\mathcal{B}_\meandiff$ is the modified
Bessel function of the second kind and order $\meandiff$,
$\sigma^2$ is the
marginal variance, $\deltah>0$ is a scaling parameter, and $\meandiff>0$
controls the mean-square differentiability of the underlying statistical process
described by the Mat\'ern covariance.
The reason for defining symbols with a hat ($\hat{\cdot}$), will become clear
in the next subsection.
Throughout, we refer to a ``Mat\'ern field'' as any Gaussian field that has
covariance that can be described by the Mat\'ern covariance function,
equation (\ref{eq:matern_covariance_iso}).

The key relationship discussed in \citet{RSSB:RSSB777} is that any Mat\'ern field,
$\unih(\xh)$, is a solution to the elliptic SPDE:
\begin{linenomath}\begin{equation}
    \Big(\deltah - \nablah\cdot\nablah\Big)^{\spdesqo/2}\hat{\uni}(\xh) =
    \Wh(\xh) \, .
    \label{eq:spde_iso}
\end{equation}\end{linenomath}
Here $\spdesqo = \meandiff + \materndim/2$,
$\Wh$ is a white noise process defined on the space $\defdomain$.
We note that the Mat\'ern covariance function describes covariances that are
stationary and isotropic.
That is, stationarity implies that correlation length scales are determined
purely by the Euclidean distance between two points and this does not change as
a function of location in the domain.
Isotropy implies that correlation lengths are the same for the same Euclidean
distance along any dimension.

The original connection between the Mat\'ern covariance function and solutions
to equation (\ref{eq:spde_iso}) was proven by
\cite{whittle_stationary_1954,whittle1963stochastic}, who
used the spectral properties of the operator $(\deltah -
\nablah\cdot\nablah)^{\spdesqo/2}$ to show that Mat\'ern fields are the only
stationary solutions to equation (\ref{eq:spde_iso}).
The result shown in \citet{RSSB:RSSB777} is
that there is an explicit link between discrete solutions to equation
(\ref{eq:spde_iso}) for any triangulation or rectangular lattice of $\ndspace$
and Mat\'ern class Gaussian fields.
The punch line is that we can use all of the computational tools for solving discretized
elliptic equations to apply a covariance operator that is formally dense.
More importantly, \citet{RSSB:RSSB777} showed that the SPDE form allows
one to easily describe Gaussian fields with more general covariance structures.
For instance, by allowing the parameter $\deltah$ to vary in space, the solution
becomes nonstationary and the Mat\'ern covariance applies locally.\\

\noindent\textbf{A more general covariance model. }
In \citet{RSSB:RSSB777} it is suggested that solving the SPDE in a transformed
coordinate system can allow for a Mat\'ern class covariance
model that can easily incorporate anisotropy and nonstationarity.
Consider the isotropic and stationary case, described by equation
(\ref{eq:spde_iso}).
The field $\unih(\xh)$ is defined in a transformed, or ``deformed''
\citep{sampson_nonparametric_1992}, space $\defdomain$.
Assume that we have a mapping $\defmap$ that maps between this transformed space
and our computational domain, $\domain$:
\begin{linenomath*}\begin{equation*}
    \defmap : \defdomain\ni\xh \rightarrow \x\in\domain \, .
\end{equation*}\end{linenomath*}
With this mapping, we can employ a change of variables
\citep{smith_change_1934} to rewrite the SPDE in the computational domain as:
\begin{linenomath*}\begin{equation*}
    \dfrac{1}{\defdet}
    \left(\deltah -
    \defdet\nabla\cdot
    \dfrac{\defjac(\x)\defjac(\x)^T}{\defdet}
    \nabla\right)\uni(\x) =
    \defdet^{-1/2}\W(\x) \, .
\end{equation*}\end{linenomath*}
Here we have defined the Jacobian as
\begin{linenomath*}\begin{equation*}
    \defjac(\x_0) \coloneqq
    \dfrac{\partial \defmap}{\partial \xh}\Big|_{\defmap^{-1}(\x_{0})} \, ,
\end{equation*}\end{linenomath*}
and for now we assume that $\defmap^{-1}(\x_0)$ is well defined.
For our purposes, this turns out to be the case, but this becomes clear when
$\defjac$ is defined in section \ref{sec:matern_operator}.
Notice that we have taken the exponent $\spdesqo/2$ to be 1, avoiding
fractional or higher order operations for simplicity.
All of the future formulations and experiments will make this assumption,
although this can be relaxed in future work.
With the following definitions:
\begin{linenomath}\begin{equation}
    K(\x) \coloneqq
    \dfrac{\defjac(\x)\defjac(\x)^T}{\defdet}
    \qquad
    \delta(\x) \coloneqq \dfrac{\deltah}{\defdet}
    \label{eq:matern_definitions}
\end{equation}\end{linenomath}
the SPDE in the computational domain's coordinate system can be written as
\begin{linenomath}\begin{equation}
    \Big(\delta(\x)- \nabla\cdot K(\x)\nabla\Big)\uni(\x) =
    \defdet^{-1/2}\W(\x) \, .
    \label{eq:spde_general}
\end{equation}\end{linenomath}
We note as in \citet{RSSB:RSSB777} that this reproduces the deformation method
introduced in \citet{sampson_nonparametric_1992}.
The key feature of this formulation is that the deformation map, $\defmap$, is
not actually necessary, only its Jacobian.
The question then becomes, how does one specify $\defjac$ and $\deltah$?
This is the primary question we wish to address.
However, we first develop the discretized form of this SPDE
that is relevant to the finite volume grid of our computational model, the
MITgcm.


\section{A Nonstationary and Anisotropic Mat\'ern Correlation Operator}
\label{sec:matern_operator}

Here we propose to use the SPDE operator described by \citet{RSSB:RSSB777}
as a means to describe an anisotropic, nonstationary correlation model in a
similar manner to the diffusion-based methods described in
\cref{ssec:wc01_review}.
To do so, we employ the ``mapping method'' described by \citet{RSSB:RSSB777}
which we show for the general $M$th order SPDE in \cref{ssec:mapping_method}.
We then provide suggestions on how to parameterize the model for intuitive use
in practical applications in \cref{ssec:scaling_laplacian,ssec:range_parameter}.


\subsection{Mapping Method via Change of Variables}
\label{ssec:mapping_method}

In \citet{RSSB:RSSB777} it is suggested that solving the SPDE in a transformed
coordinate system can allow for a Mat\'ern class covariance
model that can easily incorporate anisotropy and nonstationarity.
To do so, consider solutions to the isotropic and stationary SPDE
(\cref{eq:spde_iso}), $\unih(\xh)$, to be defined in a transformed, or
``deformed'' \citep{sampson_nonparametric_1992}, space $\defdomain$.
Then, assume that we have a mapping $\defmap$ that maps between this transformed space
and our computational domain, $\domain$:
\begin{linenomath*}\begin{equation*}
    \defmap : \defdomain\ni\xh \rightarrow \x\in\domain \, .
\end{equation*}\end{linenomath*}
With this mapping, we can employ a change of variables
\citep{smith_change_1934} to rewrite the SPDE in the computational domain as:
\begin{linenomath*}\begin{equation*}
    \dfrac{1}{\defdet^M}
    \left(\deltah -
    \defdet\nabla\cdot
    \dfrac{\defjac(\x)\defjac(\x)^T}{\defdet}
    \nabla\right)^M\uni(\x) =
    \defdet^{-1/2}\W(\x) \, .
\end{equation*}\end{linenomath*}
Here we have defined the Jacobian as
\begin{linenomath*}\begin{equation*}
    \defjac(\x_0) \coloneqq
    \dfrac{\partial \defmap}{\partial \xh}\Big|_{\defmap^{-1}(\x_{0})} \, ,
\end{equation*}\end{linenomath*}
and for now we assume that $\defmap^{-1}(\x_0)$ is well defined.
For our purposes, this turns out to be the case, but this becomes clear when
$\defjac$ is defined in \cref{sec:matern_operator}.
%Notice that we have taken the exponent $M=1$, avoiding
%fractional or higher order operations for simplicity.
%All of the future formulations and experiments will make this assumption,
%although this can be relaxed in future work.
With the following definitions:
\begin{linenomath*}\begin{equation}
    K(\x) \coloneqq
    \dfrac{\defjac(\x)\defjac(\x)^T}{\defdet}
    \qquad
    \delta(\x) \coloneqq \dfrac{\deltah}{\defdet}
    \label{eq:matern_definitions}
\end{equation}\end{linenomath*}
the SPDE in \cref{eq:spde_iso} can be written in the computational domain's coordinate system as
\begin{linenomath*}\begin{equation}
    \Big(\delta(\x)- \nabla\cdot K(\x)\nabla\Big)^M \uni(\x) =
    \defdet^{-1/2}\W(\x) \, .
    \label{eq:spde_general}
\end{equation}\end{linenomath*}
We note as in \citet{RSSB:RSSB777} that this reproduces the deformation method
introduced in \citet{sampson_nonparametric_1992}.

Here, we propose to use this generic form to define a square root of the
correlation matrix in a similar fashion to
\citet{weaver_correlation_2001, mirouze_representation_2010,
carrier_background-error_2010}
as follows,
\begin{linenomath*}\begin{equation}
    \corrMat^{1/2} \coloneqq \normalizer
    \Big(\delta(\x)- \nabla\cdot K(\x)\nabla\Big)^{-M}
    \defdet^{-1/2} \, ,
    \label{eq:matern_operator}
\end{equation}\end{linenomath*}
where $\normalizer$ is
once again a variance-preserving normalization matrix defined by the operations
that precede it.
In this model,
anisotropy and nonstationarity are controlled by
$\defjac(\x)$ and $\deltah$, and in the following subsections we discuss how these can be assigned for
practical applications in geophysical inverse problems.
We note that in this discussion we loosely mix the use of
finite dimensional matrices and infinite dimensional operators
in order to ease the presentation, but we provide a more careful
derivation of their discretized forms in \cref{sec:discretization_matern}.

\subsection{Scaling the Laplacian Term}
\label{ssec:scaling_laplacian}

First, we address the Jacobian, $\defjac(\x)$, which appears in the tensor
$K(\x)$.
We illustrate how this operator controls correlation length scales and
motivate its parameterization with a simple scaling analysis.
Consider a 3D field $\uni(\x)\sim U$ that exhibits spatial variability at the
length scales, $L_x$, $L_y$, and $L_z$ in the direction of longitude, latitude,
and height, respectively,
where $L_x, L_y >> L_z$, such that the field exhibits highly
anisotropic fluctuations.
This is a common situation in large scale geophysical fluid
dynamics, where fields (e.g.\ temperature, velocity) exhibit length scales of
variability that are much greater in either horizontal dimension compared to the
vertical.
\footnote{
    We make a note on the terminology used here.
    In oceanography, the difference in horizontal and vertical scales
    is a result of the small aspect ratio, or the shallow fluid nature of the ocean.
    Because of the vast difference in scales, the horizontal dimensions are
    sometimes considered entirely independent of the vertical, and ``anisotropy''
    can be used to refer to heterogeneity between the horizontal components.
    However, here we use anisotropy to refer to the difference between horizontal
    and vertical scales, resulting from the small aspect ratio \citep{vallis2006}.
}

Without any rescaling, i.e.\ without $K$,
the Laplacian term in $\maternOp$ is imbalanced
\begin{linenomath*}\begin{equation}
    \begin{aligned}
        \nabla^2 \uni(\x)
            & \sim \dfrac{U}{L_x^2} + \dfrac{U}{L_y^2} + \dfrac{U}{L_z^2} \\
            & \simeq \dfrac{U}{L_z^2} \, .
    \end{aligned}
    \label{eq:iso_lap}
\end{equation}\end{linenomath*}
As a result, the correlation model will
have unrealistically large or small correlations in the horizontal or vertical.
Our goal is therefore to define the elements of $K$ such that each term is of
the same order of magnitude and
\begin{linenomath*}\begin{equation*}
    \nabla\cdot K\nabla \sim 3U \, .
\end{equation*}\end{linenomath*}

To achieve this balance between Laplacian terms, we suggest a straightforward,
perhaps obvious, specification of $\defjac$:
\begin{linenomath*}\begin{equation*}
    \defjac =
        \begin{pmatrix}
            L_x & 0 & 0     \\
            0 & L_y & 0     \\
            0 & 0   & L_z   \\
        \end{pmatrix} \, ,
\end{equation*}\end{linenomath*}
where we simply ignore the off-diagonal elements of $\defjac$.
The determinant in this case is $\defdetnox = L_xL_yL_z$ and
according to the definitions in \cref{eq:matern_definitions}:
\begin{linenomath*}\begin{equation*}
    K =
        \begin{pmatrix}
            1/L_z & 0 & 0     \\
            0 & 1/L_z & 0     \\
            0 & 0   & L_z/(L_xL_y)   \\
        \end{pmatrix} \, ,
\end{equation*}\end{linenomath*}
so that
\begin{linenomath*}\begin{equation*}
    \nabla\cdot K(\x)\nabla\uni(\x) \sim \dfrac{3}{L_xL_yL_z}U \, .
\end{equation*}\end{linenomath*}
The key is that $K$ scales each term in the Laplacian so that they are
approximately the same order of magnitude, and the operator is balanced in either direction.
%This example can further be extended by allowing the length scales $L_1$ and
%$L_2$ to vary in space, such that correlations are nonstationary.

At this point, we must prescribe values for $L_x(\x)$, $L_y(\x)$, and $L_z(\x)$ to fill
$\defjac(\x)$.
Considering the discretization of the Laplacian, a simple choice for these can
be based on the underlying grid of the general circulation model.
For the numerical experiments in this paper, we choose
$L_x(i,j) = \Delta x_g(i,j)$, $L_y(i,j) = \Delta y_g(i,j)$,
and $L_z(k) = \Delta r_f(k)$ (\cref{fig:mitgcm_grid}) where we have switched
from the spatial coordinate $\x\in\domain$ to the computational grid indices $i$, $j$, $k$.
We consider the choice to use the grid elements directly to be reasonable
because anisotropy and nonstationarity is usually encoded into the grid.
For instance, regarding nonstationarity, it is reasonable to assume that
correlation length scales near the surface of the ocean are relatively short as
a result of the locality of atmosphere-ocean interactions that occur there.
On the other hand, near the ocean floor motion is presumed to be more
quiescient, and correlation length scales are correspondingly longer.
Many ocean models
\citep[e.g.][]{nguyen_arctic_2021, forgetECCOv4}
incorporate these assumptions into the vertical grid resolution.


\subsection{The Range Parameter}
\label{ssec:range_parameter}

The other term in \cref{eq:spde_general} to be defined is $\deltah$.
Here we resort to the empirical relation suggested in
\citet{RSSB:RSSB777}:
\begin{linenomath*}\begin{equation*}
    \rangeh = \sqrt{\dfrac{8\meandiff}{\deltah}} \, .
\end{equation*}\end{linenomath*}
The so-called range parameter, $\rangeh$, defines the
distance between two points at which correlation drops to 0.14.
We note that the range parameter used here is larger by a factor of two than
what is commonly often used, for instance this is twice the scale parameter used
by \citet{mirouze_representation_2010}.
We prefer the definition used here because the relation described above provides an intuitive
understanding of $\rangeh$.

%While it may seem like we have swapped one unknown for another, it is usually easier to
%define correlation length scales than simply guessing values for $\deltah$.
With this relation and our choice for $\defjac(\x)$ in
\cref{ssec:scaling_laplacian} we have
\begin{linenomath*}\begin{equation*}
    \begin{aligned}
        \delta(i,j,k) &= \dfrac{8\meandiff}{\rangeh^2\defdetd} \\
                      &= \dfrac{8\meandiff}{\rangeh^2 L_x(i,j,k)L_y(i,j,k) L_z(i,j,k)} \, .
    \end{aligned}
\end{equation*}\end{linenomath*}
With this choice, both this term and the Laplacian term are now of a similar order of
magnitude
\begin{linenomath*}\begin{equation*}
    D_{\delta}\unis \sim \nabla \cdot K \nabla \unis \sim
    \bigo\left(\dfrac{\meandiff}{L_xL_yL_z}\right) \, ,
\end{equation*}\end{linenomath*}
where we note that $\meandiff \sim \bigo(1)$ is a reasonable assumption since
$\meandiff=5/2 \implies M=2$ gives a good approximation to a Gaussian
correlation when $d=3$, see \red{FIG}.

With $L_x$, $L_y$, and $L_z$ defined by the model grid elements, the elliptic
operator in \cref{eq:matern_operator} is effectively nondimensionalized.
We assert that with this definition, one can use this operator on an
anisotropic and spatially varying computational grid armed with the intuition that
$\rangeh$ is a nondimensional parameter that controls the
``number of neighboring grid cells'' at which correlation decays to 0.14.
Our numerical experiments in \cref{sec:llc90} show that this is a good
approximation in the case of a realistic global ocean model grid.



\section{Application to the Global Ocean}
\label{sec:llc90}

Here we describe the LLC90 grid

\noindent\textbf{Samples from the covariance model.}

%\begin{figure}
%    \centering
%    \includegraphics[width=\textwidth]{../figures/samples_and_pointwise_std.jpg}
%    \caption{Normalized samples and pointwise standard deviation from the covariance
%        model.
%        (top row) Samples from the covariance model, normalized by the
%        pointwise standard deviation. Each column shows increasing correlation
%        length scales, which is regulated by $\rangeh$.
%        The arrows in the bottom right corner of each plot show the approximate length scales
%        that are covered by $\rangeh L_y$ and $\rangeh L_z$, respectively.
%        Here we have used $L_y = 2\Delta y_g$ and $L_z=\Delta r_f$, see
%        Figure \ref{fig:mitgcm_grid} for a notational reference.
%        We note that the standard normal vector $\mathbf{z}$ is unique in
%        each figure.
%        (bottom row) Pointwise standard deviation of the covariance operator
%        $CC^T$, estimated from
%        a sample size of $N=1000$. The diagonal matrix $X$ is comprised of the
%        inverse of this field.
%    }
%    \label{fig:matern_samples}
%\end{figure}


\noindent\textbf{Correlation length scales.}

Here we do not use the Jacobian, but just compute based on neighboring grid
cells.
\begin{figure}
    \centering
    \includegraphics[width=\textwidth]{../figures/llc90_nondimensional_correlation_k.pdf}
    \caption{Correlation coefficient using the LLC90 (global ocean) grid, with
        distances computed in the vertical dimension from 100 samples. The coefficient is
        computed from the random samples used to estimate
        $\hat{\sigma}$ and $X$. Each color denotes correlations computed for a
        different value of $\rangeh$, and the spread is determined by one standard
        deviation around the domain averaged correlation coefficent.
        The black curve denotes the predicted isotropic correlation predicted by
        equation \eqref{eq:matern_correlation_iso} with the appropriate value for
        $\rangeh$.}
    \label{fig:llc90_correlations}
\end{figure}


\section{Discussion}
\label{sec:matern_discussion}

In this work we have shown a general methodology for applying a Mat\'ern type
correlation function within a domain with complex boundaries, while achieving
nonstationary and anisotropic correlation length scales.
To summarize, the general procedure is as follows.
First, one chooses a \red{normalization length scale} for each dimension, thereby
defining (the Jacobian of) a mapping between a space where correlation is
isotropic and stationary, and the more complex domain with nonstationary and
anisotropic correlation length scales.
Next, one must choose a range parameter, determining the distance relative to
the \red{normalizing length scales} at which correlation drops to 0.14.
Finally, one selects the shape of the correlation structure, which also sets the
number of times the elliptic PDE must be solved.

Our presentation has focused on the practical application of this
correlation operator within an ocean general circulation model.
As such, we set the \red{normalizing length scales} based on the local grid
scale.
With this setting, the range parameter is shown to be a highly intuitive dial,
controlling correlation length scales as a simple function of the number of
neighboring grid cells.
Using this definition was further shown to be beneficial at the equator, for
example, because grid scale refinements there result in relatively
longer zonal than meridional correlation length scales and this is \red{observed
to be physically realistic}.
However, we recognize that there could be features that are desirable to capture
in a correlation model that are not represented in the definition of the
underlying model grid.
In this case, the \red{normalizing length scales} could be further tuned with
local factors or functions to achieve these desired features.
Alternatively, the \red{length scales} could be set entirely independently of
the grid, for instance as a function of the local Rossby radius of deformation.

A key feature of the correlation model shown here is the $\rangeh$ and $M$
control the correlation length scale and shape \textit{separately}.
We consider this to be an attractive feature when compared to the implicit
diffusion approach \red{albeit, without changing the order of the laplacian or
any parameters as described in} \citet{weaver_diffusion_2013}.
In this model, at a given ``length scale'' the distance at which correlation
drops to a specified threshold still changes with $M$
(\cref{fig:correlation_comparison}, and also
\citep[Figs. 1 and 2 of][]{guillet_modelling_2019}).
We note that in the Mat\'ern correlation model presented here that the parameter
$\deltah$ changes with $M$ while in the implicit diffusion approach, $\deltah
\rightarrow 1$.
Apparently the simple variation in this parameter is enough to balance the
multiple applications of $\maternop$, such that the resulting correlation
structure maintains a consistently identifiable length scale via $\rangeh$.

As noted in \citet{mirouze_representation_2010,carrier_background-error_2010}, a
drawback to the explicit diffusion approach from
\citet{weaver_correlation_2001} is that it requires many iterations to satisfy
numerical stability.
We note specifically in our experimentation with this model on the LLC grid
as implemented in the MITgcm \citep{campin_mitgcmmitgcm_2021}, we have found
the number of iterations required for numerical stability to be roughly a factor
of three larger than the necessary (but insufficient) bound for numerical
stability.
We therefore find approaches based on the implicit solution of a PDE to be more
straightforward, as it is more intuitive to specify a solution tolerance
rather than guess the number of iterations required for convergence.
Moreover, our numerical experiments indicate that the solution can be imprecise
(to a tolerance of $\sim10^{-3}$) and therefore highly efficient.
Finally, because the correlation model shown here is formulated through an
inverse elliptic operator, we have access to the inverse correlation operator,
which could be used directly as regularization while solving an inverse problem
\citep[e.g.][]{bui-thanh_computational_2013}, or for the specification of
spatially correlated observations as in \citet{guillet_modelling_2019}.

For some applications it could be desirable to specify oscillating or ``lobed''
correlation models, which can be achieved with the explicit or implicit
diffusion models \citep{weaver_correlation_2001,weaver_diffusion_2013}.
We suggest that such extensions are possible for the Mat\'ern type correlation
operator shown here, based on results shown by \citet{RSSB:RSSB777} in the
complex plane with a tunable oscillation parameter.
These more general shapes could be explored in future work for the case of
multi-dimensional fields as shown here.


\appendix
\section{Discretization of the Mat\'ern SPDE}
\label{sec:discretization_matern}

In the following analysis we consider a 2D field
$\uni(\x)$, $\x\in\openBdy$ since this is
relevant to our control parameters, although we note that the
extension to 3D is relatively straightforward.
We carry out the discretization on a structured nonuniform grid according to the
finite volume method - as is the general setting in the MITgcm.
We note that our development is similar to \citet{fuglstad_exploring_2015},
who show a differential operator for a 2D field on a uniform grid.

Figure \ref{fig:mitgcm_grid} shows the general structure of the grid, and
defines the various grid cell distances used in the derivation.
We begin by integrating equation (\ref{eq:spde_general}),
\begin{linenomath}\begin{equation}
    \begin{aligned}
        \int_{\openBdy} \delta(\x)\,\uni(\x)\, d\x -
        \int_{\openBdy} \nabla\cdot K(\x)\nabla\,\uni(\x)\,d\x
        &=
        \int_{\openBdy} \W(\x)\defdet^{-1/2} \, d\x \\
        \sum_{\jk}\int_{\cell_\jk} \delta(\x)\,\uni(\x)\, d\x -
        \sum_{\jk}\int_{\cell_\jk} \nabla\cdot K(\x)\nabla\,\uni(\x)\,d\x
        &=
        \sum_{\jk}\int_{\cell_\jk} \W(\x)\defdet^{-1/2} \, d\x \, ,
    \end{aligned}
    \label{eq:spde_integral}
\end{equation}\end{linenomath}
where in second line we distribute the integral across each grid cell
$\cell_\jk\subset\openBdy$.

\begin{figure}
    \centering
    \includegraphics[width=.6\textwidth]{../figures/hgrid-abcd.pdf}
    \includegraphics[width=.2\textwidth]{../figures/vgrid-accur-center.pdf}
    \caption{The structured finite volume grid used in the MITgcm. The
        left two figures show the horizontal grid, viewed from above. The right
        figure shows the vertical grid. In each figure, u, v, and w mark the
        location where velocities exist on the grid cell interfaces. The open
        circles denote the tracer location at the grid cell center,
        where fields like temperature and salinity are located. Figures are from
    \citet{campin_mitgcmmitgcm_2021}.}
    \label{fig:mitgcm_grid}
\end{figure}

Starting with the first term,
\begin{linenomath*}\begin{equation*}
    \delta_\jk \coloneqq \dfrac{1}{\vol_\jk}\int_{\cell_\jk}\delta(\x)\,d\x
\end{equation*}\end{linenomath*}
so that
\begin{linenomath*}\begin{equation*}
    \int_{\cell_\jk} \delta(\x)\,\uni(\x)\, d\x =
    \vol_\jk\,\delta_\jk\,\uni_\jk
\end{equation*}\end{linenomath*}
where $\vol_\jk = \Delta y^{j}_{g} \Delta r_f^k$ is the grid cell volume,
indicated by Figure \ref{fig:mitgcm_grid}.
We note that a cartesian style notation
is used, but spherical polar coordinates are used in the computation, and the
MITgcm supports generic curvilinear coordinates.
Generally, the coordinate directions $(x,y,z)$ correspond to $(\lambda,\phi,r)$,
i.e. longitude, latitude, and height.
In the spherical polar case, the differential elements are
\begin{linenomath*}\begin{equation*}
    \Delta x = r_0 \cos(\phi)\Delta\lambda \qquad
    \Delta y = r_0 \Delta \phi \qquad
    \Delta z = \Delta r
\end{equation*}\end{linenomath*}
where $r_0 = 6,378$ km is the nominal radius of the earth, and note that in this
coordinate system latitude $\phi$ is defined to be 0 at the equator.

The third term is, from the definition of a white noise process
\citep{adler_random_2007},
\begin{linenomath*}\begin{equation*}
    \int_{\cell_\jk}\defdet^{-1/2}\W(\x) \, d\x
        = \sqrt{\dfrac{\vol_\jk}{\defdetd}} z_\jk
\end{equation*}\end{linenomath*}
where $z_\jk$ is an uncorrelated (independent) standard Gaussian at each grid
cell center $\jk$, and we used:
\begin{linenomath*}\begin{equation*}
    \defdetd \coloneqq
    \dfrac{1}{\vol_\jk}\int_{\cell_\jk}\defdet\,d\x\,.
\end{equation*}\end{linenomath*}

The second term, containing the Laplacian is handled as follows
\begin{linenomath*}\begin{equation*}
    \int_{\cell_\jk} \nabla\cdot K(\x)\nabla\,\uni(\x)\,d\x
    =\int_{\cellbdy_\jk} K(\x)\nabla\,\uni(\x)\,
    \cdot \hat{\mathbf{n}} \, d\s
\end{equation*}\end{linenomath*}
where $\hat{\mathbf{n}}$ is an outward normal to the cell boundary
$\cellbdy_\jk$.
Throughout this work, we assume the tensor $K(\x)$ to be represented as the
diagonal matrix:
\begin{linenomath*}\begin{equation*}
    K(\x) =
    \begin{pmatrix}
        \kappa^{vy}(\x) & 0 \\
        0 & \kappa^{wz}(\x) \\
    \end{pmatrix} \, .
\end{equation*}\end{linenomath*}
Our choice here is discussed later, and could be generalized in future work.
We represent the
discretized form of this tensor as
\begin{linenomath*}\begin{equation*}
    K_\jk =
    \begin{pmatrix}
        \kappa^{vy}_\jk & 0 \\
        0 & \kappa^{wz}_\jk \\
    \end{pmatrix} \, ,
\end{equation*}\end{linenomath*}
where the elements $\kappa^{vy}_\jk$ and $\kappa^{wz}_\jk$
are located at the $v$ and $w$ grid cell locations in Figure
\ref{fig:mitgcm_grid}.
Additionally
\begin{linenomath*}\begin{equation*}
    \kappa^{vy}_{\jk} \coloneqq \dfrac{1}{\Delta r_f^{\jk}}
    \int_{\cellbdy_{\jk}^{S}} \kappa^{vy}(\x)\,d\x
    \qquad
    \kappa^{wz}_{\jk} \coloneqq \dfrac{1}{\Delta y_g^{\jk}}
    \int_{\cellbdy_{\jk}^{B}} \kappa^{wz}(\x)\,d\x
\end{equation*}\end{linenomath*}
where $\cellbdy_{\jk}^{S}$ and $\cellbdy_{\jk}^{B}$ are the
southern and bottom boundaries of the grid cell.
The discretized gradient is approximated via the finite difference
directional derivative at each cell face:
\begin{linenomath*}\begin{equation*}
    \begin{aligned}
%        \pderiv{\uni}{x}(\x_\jk^W)
%        &\simeq \dfrac{\uni_\jk - \uni_\imjk}{\Delta x_c^\jk}
%        \qquad
%        \pderiv{\uni}{x}(\x_\jk^E)
%        &&\simeq \dfrac{\uni_\ipjk - \uni_\jk}{\Delta x_c^\ipjk}
%        \\
        \pderiv{\uni}{y}(\x_\jk^S)
        &\simeq \dfrac{\uni_\jk - \uni_\jmk}{\Delta y_c^\jk}
        \qquad
        \pderiv{\uni}{y}(\x_\jk^N)
        &&\simeq \dfrac{\uni_\jpk - \uni_\jk}{\Delta y_c^\jpk}
        \\
        \pderiv{\uni}{z}(\x_\jk^B)
        &\simeq \dfrac{\uni_\jk - \uni_\jkm}{\Delta z_c^\jk}
        \qquad
        \pderiv{\uni}{z}(\x_\jk^T)
        &&\simeq \dfrac{\uni_\jkp - \uni_\jk}{\Delta z_c^\jkp}
    \end{aligned}
\end{equation*}\end{linenomath*}
for the south, north, bottom, and top cell faces, respectively.
Putting these definitions together,
\begin{linenomath}\begin{equation}
    \begin{aligned}
        \int_{\cellbdy_\jk}
        &K(\x)\nabla\,\uni(\x)\,
        \cdot \hat{\mathbf{n}} \, d\s
        \coloneqq \\
%        &\left[
%            \left(
%            \dfrac{\kappa^{ux} \, \Delta y_g \Delta r_f}{\Delta x_c}
%            \right)_\ipjk \,
%            (\uni_\ipjk - \uni_\jk) -
%            \left(
%            \dfrac{\kappa^{ux} \, \Delta y_g \Delta r_f}{\Delta x_c}
%            \right)_\jk \,
%            (\uni_\jk - \uni_\imjk)
%        \right]+
%        \\
        &\left[
            \left(
            \dfrac{\kappa^{vy} \, \Delta r_f}{\Delta y_c}
            \right)_\jpk \,
            (\uni_\jpk - \uni_\jk) -
            \left(
            \dfrac{\kappa^{vy} \, \Delta r_f}{\Delta y_c}
            \right)_\jk \,
            (\uni_\jk - \uni_\jmk)
        \right]+
        \\
        &\left[
            \left(
            \dfrac{\kappa^{wz} \, \Delta y_g}{\Delta r_c}
            \right)_\jkp \,
            (\uni_\jkp - \uni_\jk) -
            \left(
            \dfrac{\kappa^{wz} \, \Delta y_g}{\Delta r_c}
            \right)_\jk \,
            (\uni_\jk - \uni_\jkm)
        \right]
        \, .
    \end{aligned}
    \label{eq:big_laplacian}
\end{equation}\end{linenomath}
With each term in equation \eqref{eq:spde_integral} defined above, we have the system of
equations in matrix form:
\begin{linenomath}\begin{equation}
    \begin{aligned}
        (D_\delta - L) \unis &= D_z\mathbf{z}\\
        A\unis &= D_z\mathbf{z}
    \end{aligned}
    \label{eq:fv_spde}
\end{equation}\end{linenomath}
where we have (left) divided by grid cell volume so that:
\begin{linenomath*}\begin{equation*}
    \begin{aligned}
        D_\delta \coloneqq \text{diag}\{\delta_i\}_{i=1}^{\nuni} \qquad
        D_z \coloneqq \text{diag}\left\{
            \dfrac{1}{\sqrt{\vol_i \,\, \defdetdi}}
            \right\}_{i=1}^{\nuni}
    \end{aligned}
\end{equation*}\end{linenomath*}
where for notational simplicity we index each grid cell with $i$, rather than
$j$ and $k$ as above.
Finally, $L$ is defined by appropriately gathering terms in
equation \eqref{eq:big_laplacian}, applying the boundary conditions discussed in the next
subsection, and left dividing by grid cell volume.

At last, the covariance matrix associated with the vector $\unis$ can be
obtained by considering the joint probability distribution of
$\mathbf{z}\sim\mathcal{N}(0,I)$ and
the relation $\mathbf{z} = D_z^{-1}A\unis$, obtained from
equation \eqref{eq:fv_spde}.
That is,
\begin{linenomath*}\begin{equation*}
    \begin{aligned}
        \pi(\mathbf{z})
        &\propto  \exp\left\{-\dfrac{1}{2}\mathbf{z}^T\mathbf{z}\right\} \\
        &\propto \exp\left\{-\dfrac{1}{2}\unis^T A D_z^{-2} A \unis\right\} \\
        &\propto \pi(\unis) \, ,
    \end{aligned}
\end{equation*}\end{linenomath*}
where we note that $A=A^T$.
Thus, we define the differential operator for our covariance model as
\begin{linenomath*}\begin{equation*}
    C \coloneqq A^{-1} D_z \, .
\end{equation*}\end{linenomath*}
\\

\noindent\textbf{Neumann boundary conditions.}
So far we have omitted the discussion of boundaries for the sake of
simplicity.
We employ the Neumann boundary condition:
\begin{linenomath*}\begin{equation*}
    K(\x)\nabla\uni(\x) \cdot \hat{\mathbf{n}} = 0 \qquad \x\in\groundBdy \, .
\end{equation*}\end{linenomath*}
which are implemented in equation \eqref{eq:big_laplacian}
simply by zeroing out the gradient term at solid boundaries.
We note that this has an impact on the covariance structure that is discussed
more concretely with the numerical results in
\red{section XX}.\\

\noindent\textbf{Generic form of the covariance.}
With the Mat\'ern covariance operator $C$ defined, we recall that
\begin{linenomath*}\begin{equation*}
    \Gamma_{\uni} = \Gamma_{\uni}^{1/2}\Gamma_{\uni}^{T/2}
    = \Sigma X C C^T X \Sigma \, ,
\end{equation*}\end{linenomath*}
and all that is left to define are the operators $X$ and $\Sigma$.
We reserve the definition of $\Sigma$ to
chapter xx, as this becomes somewhat specific to our
application.
Recall that $X$ is defined through the pointwise marginal variance of the
covariance operator $C$.
This can feasibly be computed as $\mathbf{e}_i^TCC^T\mathbf{e}_i$, with
$\mathbf{e}_i$ being the canonical basis vector associated with the $i$th
element of the grid.
However, we compute an approximation based on samples from the covariance model, as
suggested in \citet{weaver_correlation_2001} (and references therein).
Samples of the Gaussian random variable with mean $\unis_0$ and covariance
$C C^T$ are computed as
\begin{linenomath*}\begin{equation*}
    \unis = \unis_0 + C\mathbf{z} \, .
\end{equation*}\end{linenomath*}
We find $X$ by computing the standard deviation from a sample size of $N=1000$.
We note that using this approach to approximate the variance of $C$ scales well
to higher dimensional applications.
Additionally, this approach fits well into our computational framework, since
computing these random samples is a necessary first step for the randomized
eigenvalue decomposition algorithm that is fundamental to our inference problem.


\section{Appendix A: A Block Successive Over Relaxation Method}
\label{sec:block_sor}

Applying the Mat\'ern type prior covariance operator, $C$
requires the solution to an elliptic equation.
We outline here a block-Successive Over Relaxation (SOR) method that was
implemented to solve this problem.
We first make a few notes regarding why we chose this algorithm, and why it had
to be implemented at all.

For better or for worse, the MITgcm is a standalone package, and any code
modifications must not rely on outside packages in order to be incoroporated in
the main distribution.\footnote{The notable exception being the expensive,
    proprietary algorithmic differentiation software TAF \citep{giering2005}
    that is tightly interwoven into the code to enable the adjoint model.}
Therefore, incorporating a solver from e.g.\ \texttt{PETSc} is not allowed.
The MITgcm does contain a conjugate gradient solver, but this relies on
a preconditioning that is motivated by geophysical fluid dynamics that may not
be relevant to the more general Mat\'ern SPDE structure.
Thus, we use the SOR method as it is generally easy to implement, and is ``fast
enough'' noting that the computational cost of most elliptic solvers will be
inconsequential when compared to the time-dependent forward model.

The SOR method is an iterative method for solving $A\mathbf{x} = \mathbf{b}$.
At iteration $k$, the elements of $\mathbf{x}$ are $x_i^k$ (similar
notation for other variables), and we seek the update:
\begin{equation}
    \tilde{x}_i^{k+1} = (1-\omega) x_i^k + \dfrac{\omega}{a_{ii}}
    \left( b_i - \sum_{j<i}a_{ij}\tilde{x}_j^{k+1} -
        \sum_{j>i}a_{ij}x_j^{k}\right), \qquad
        i=1,2,...,N \, ,
    \label{eq:sor_update}
\end{equation}
where $\omega$ is the SOR parameter.
We compare this method to the standard Jacobi update
$$\tilde{x}_i^{k+1} = x_i^k + \sum_{i\ne j}\dfrac{a_{ij}x_j^k}{a_{ii}}, \qquad
i=1,2,...,N \, .$$
Here the notation $\tilde{x}_i^{k+1}$ refers to the fact this is a local update,
i.e. there is no communication between the processes assigned to each portion of
the computational domain.
The only areas where this local update causes the algorithm to deviate from a
true SOR method is where neighboring elements
$\tilde{x}^{k+1}_{j}, i\ne j$ are in the ``halo'' regions (i.e. outside of a
process's subdomain).
In this case, $\tilde{x}^{k+1}_j = x^k_j$.

\begin{figure}
    \includegraphics[width=\textwidth]{../figures/sor_noXi.pdf}
    \caption{Performance of the SOR algorithm for various settings of $\omega$.
        Average number of SOR or Jacobi iterations to convergence are shown
        based on solving equation \eqref{eq:fv_spde} with 1000 samples from a
        standard normal as the right hand side. Here $\xi=1$.}
    \label{fig:sor}
\end{figure}

We find this simple modification to be an effective means of speeding up the
linear solve.
Figure \ref{fig:sor} shows that the number of iterations required for
convergence is reduced to about 30\% of the Jacobi scheme.
Of course, a major drawback of the SOR algorithm is exhibited here as well: the
efficiency is highly sensitive to the parameter $\omega$, as shown in
right panel of Figure \ref{fig:sor}.
We
additionally find the performance to be dependent on the length scales used in the Jacobian
$\defjac$, defined in section \ref{sec:matern_operator}.
To facilitate the discussion, we define $L_y = \xi \Delta y$, where $\xi$ is
some multiple that accentuates length scales further in the meridional direction
(e.g.\ $\xi=2$ in many of the results shown in this work).
The vertical length scale is kept the same as before, i.e.\ $L_z = \Delta r$.
We report here that the optimal SOR parameter depends on $\xi$, as shown in
Figure \ref{fig:sor}, and
the optimal parameter for each $\xi$ tested is shown in
Table \ref{table:sor_xi}.
As $\xi$ increases, $\omega^*\rightarrow 1$, and the algorithm approaches
the standard Jacobi algorithm.

\begin{table}[]
    \centering
    \caption{Optimal SOR parameter $\omega^*$ as a function of
        $\xi$, where $L_y = \xi \Delta y$.}
    \label{table:sor_xi}
    \begin{tabular}{c|c|c|c|c}
                   & $\xi=1/2$ & $\xi=1$ & $\xi=2$ & $\xi=5$ \\ \hline
        $\omega^*$ & 1.8 & 1.6 & 1.3 & 1.2                   \\
\end{tabular}
\end{table}

Finally, we note that in the current implementation of this algorithm we update
the ``halo'' regions of $\x$, i.e. $\tilde\x\rightarrow\x$, at the end of each iteration in the
solver.
We recognize that performance could be improved further by
increasing the number of iterations taken before updating the halo regions,
in to reduce communication.
This possible performance benefit could be explored in future work.


%\section{Application to the Pine Island ice shelf domain}
\label{sec:matern_pig}

Here we show the implementation of the covariance model within the MITgcm,
using the computational grid that we will employ to study the Pine Island cavity
circulation.
We discuss more details on how the grid is obtained in
chapter xx.
However, we mention that the horizontal grid scale is
approximately $\Delta x \simeq \Delta y \simeq 600$~m, and the vertical grid
scale is $\Delta r = 20$~m.
Finally, we note that applying $C$ to a random gaussian vector $\mathbf{z}$
requires the solution to an inverse elliptic operator,
which we obtain with an implementation of a block-Successive Over Relaxation
(SOR) method.
The MITgcm uses no external packages, so interfacing with
e.g.\ \texttt{PETSc} is not an option for any code that is meant to be
contributed to the main MITgcm distribution - which is our intention.
While the solver is far from perfect, it provides decent performance and is
easy to implement.
We discuss details of the implementation and
performance in xx. \\

\noindent\textbf{Samples from the covariance model.}
Figure \ref{fig:matern_samples} (top row) shows samples from the normalized
covariance at the western open boundary of the computational domain,
$XC\mathbf{z}$.
In all cases we use $L_y = 2\Delta y$ and $L_z = \Delta r$, so correlation
length scales vary purely by the specification of $\rangeh$.
The approximate length scale associated with $\rangeh$ is indicated by the arrows on
the bottom right corner of each plot, which qualitatively compares well to the
structure of the random samples.
We note that their is high anisotropy in the domain:
the meridional extent at $Z=0$ (i.e. non-gray distance along the x-axis at the
top of each figure) is $\sim$73~km, while the maximum depth is $\sim1$~km.

The random samples in Figure \ref{fig:matern_samples} (top row)
are normalized by $X$ to have approximately unit variance.
The pointwise marginal standard deviation, $\hat{\sigma}$, which is the inverse
of the diagonal elements of $X$, is shown in Figure \ref{fig:matern_samples}
(bottom row).
With larger correlation length scales, the standard deviation of $C$ increases.
This is particularly true near the boundaries of the field, where we see narrow
bathymetric trenches extending to depth.\\

\begin{figure}
    \centering
    \includegraphics[width=\textwidth]{../figures/samples_and_pointwise_std.jpg}
    \caption{Normalized samples and pointwise standard deviation from the covariance
        model.
        (top row) Samples from the covariance model, normalized by the
        pointwise standard deviation. Each column shows increasing correlation
        length scales, which is regulated by $\rangeh$.
        The arrows in the bottom right corner of each plot show the approximate length scales
        that are covered by $\rangeh L_y$ and $\rangeh L_z$, respectively.
        Here we have used $L_y = 2\Delta y_g$ and $L_z=\Delta r_f$, see
        Figure \ref{fig:mitgcm_grid} for a notational reference.
        We note that the standard normal vector $\mathbf{z}$ is unique in
        each figure.
        (bottom row) Pointwise standard deviation of the covariance operator
        $CC^T$, estimated from
        a sample size of $N=1000$. The diagonal matrix $X$ is comprised of the
        inverse of this field.
    }
    \label{fig:matern_samples}
\end{figure}


\noindent\textbf{Correlation length scales.}
Here we show that the correlations obtained from random samples agree well with
the structure that one would expect from the Mat\'ern model.
We consider the correlation function associated with the isotropic Mat\'ern
covariance from equation \eqref{eq:matern_covariance_iso}:
\begin{linenomath*}\begin{equation*}
    \begin{aligned}
        r(\rangeh,\xh_1,\xh_2) &= r(\rangeh, ||\xh_1-\xh_2||) \\
                       &= c(\xh_1,\xh_2)/\sigma^2 \\
                       &= \dfrac{1}{2^{\meandiff-1}\mathcal{G}(\meandiff)}
        \left(\dfrac{\sqrt{8\meandiff}}{\rangeh} ||\xh_2-\xh_1||\right)^\meandiff
        \mathcal{B}_\meandiff
        \left(\dfrac{\sqrt{8\meandiff}}{\rangeh} ||\xh_2-\xh_1||\right) \, ,
    \end{aligned}
\end{equation*}\end{linenomath*}
which emphasizes that correlation is determined purely by Euclidean distance in
the transformed space.

We compare correlation distances in the computational domain to the
theoretical value, $r(\rangeh,||\xh_1-\xh_2||)$, by using the mapping method described in
section \ref{ssec:mapping_method}.
Specifically, we compute the correlation coefficient from the random samples
used to approximate $X$ as a function of distance in the computational domain.
We then map these distances back to the ``transformed'' space $\defdomain$ via
the inverse map $\defmap^{-1}$.
We note that $\defmap$ has a differentiable inverse $\defmap^{-1}$ by construction
through the inverse function theorem,
as a result of defining $\defmap$ through its Jacobian $\defjac$ such that
$\defdet \neq 0 \,\,\forall \,\,\x\in\openBdy$.
Specifically, distances in the computational domain are mapped to the
transformed space for arbritrary points $\x_1,\x_2\in\openBdy$ such that
$\xh_1\coloneqq\defmap^{-1}(\x_1)$ and $\xh_2\coloneqq \defmap^{-1}(\x_2)$ as
follows:
\begin{linenomath*}\begin{equation*}
    \begin{aligned}
        \xh_2 - \xh_1 &= \defmap^{-1}(\x_2) - \defmap^{-1}(\x_1) \\
                      &= \left(
                            \defmap^{-1}(\x_1) +
                            \defjac^{-1}\big|_{\x_{1}}(\x_2-\x_1) +
                            \bigo( ||\x_2-\x_1|| )
                        \right) - \defmap^{-1}(\x_1) \\
                        &\simeq \defjac^{-1}\big|_{\x_{1}}(\x_2-\x_1) \, .
    \end{aligned}
    \label{eq:matern_correlation_iso}
\end{equation*}\end{linenomath*}
That is, we use the inverse Jacobian to map distances in the computational domain to the
transformed space $\defdomain$, where we can compare correlation statistics to
the Mat\'ern formula.
We make one final approximation to ease the process.
In the spherical polar coordinate system used, recall that
$\Delta y = r_0 \Delta \phi$, i.e. the meridional grid spacing varies with
latitude.
However, the meridional extent of our computational domain is relatively small,
such that $\Delta y$ only varies from $\sim 582-620$~m.
We therefore assume that
$\defjac^{-1}\big|_{\x_{1}} =\defjac^{-1}\,\,\forall\,\,\x_1\in\openBdy$
simply for the sake of this calculation.

The sample correlation coefficent computed from the 1000 random samples are
shown for various values of $\rangeh$ in Figure \ref{fig:matern_correlations}.
We separately compute distances in the meridional (left panel) and vertical
(right panel) directions as
\begin{linenomath*}\begin{equation*}
    \delta \hat{y} = \defjac^{-1}\delta y
    \qquad
    \delta \hat{z} = \defjac^{-1}\delta z \, .
\end{equation*}\end{linenomath*}
Each colored curve shows the domain averaged correlation coefficient for a
particular $\rangeh$, and the spread denotes one standard deviation above and
below the mean.
Each black curve shows the theoretical, predicted correlation for the distances
in the deformed space $\delta\hat{y}$ and $\delta\hat{z}$.
At larger values of $\rangeh$, the spread in correlation coefficient and
mismatch between the theoretical prediction grow larger.
We attribute this to boundary effects, and note that this slice of the
computational domain has at its maximum extent 60~$L_y$ and 53~$L_z$.
Thus, we expect some degree of difference between the SPDE derived correlation
and the functional form of the covariance model.
However, in general we find the agreement to be quite good.
The result is that we can formulate the covariance model (i.e.\ choose
$\rangeh$) based on our intuition from the simple isotropic form, and this is
easily mapped to the computational domain.

\begin{figure}
    \centering
    \includegraphics[width=\textwidth]{../figures/nondimensional_correlation.pdf}
    \caption{Correlation coefficient corresponding to different distances in the
        (left) meridional and (right) vertical directions. The coefficient is
        computed from the random samples used to estimate
        $\hat{\sigma}$ and $X$. Each color denotes correlations computed for a
        different value of $\rangeh$, and the spread is determined by one standard
        deviation around the domain averaged correlation coefficent.
        The black curve denotes the predicted isotropic correlation predicted by
        equation \eqref{eq:matern_correlation_iso} with the appropriate value for
        $\rangeh$.}
    \label{fig:matern_correlations}
\end{figure}



\acknowledgments
Thanks everyone.

\bibliography{references}
\end{document}
