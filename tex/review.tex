\section{Mathematical Context and Background}
\label{sec:review}


\subsection{Inverse Problem Formulation}
\label{ssec:da_formulation}

We consider the general problem of finding the optimal control vector,
$\params$, which minimizes the regularized model-data misfit cost function
\begin{linenomath*}\begin{equation*}
    \cf(\params) =
        \dfrac{1}{2}||\pto(\params) - \data||_{\obsCovMat^{-1}}^2
        +
        \dfrac{1}{2}||\params - \priorParams||_{\priorCovMat^{-1}}^2 \, .
\end{equation*}\end{linenomath*}
The solution to this inverse problem, $\paramsMAP$, arises from a tradeoff between fitting the
observational data, $\data$, as governed by the parameter to observable map
$\pto(\cdot)$, and minimizing deviations from the background-state $\priorParams$.
On the one hand, the observational error covariance matrix
$\obsCovMat$ represents our uncertainty
in the observational data, together with our confidence in the model's ability
to represent the observed values \red{CITE REPRESENTATIVENESS}.
On the other hand, the background-state covariance matrix $\priorCovMat$
represents our uncertainty in the prior estimate or background state,
$\priorParams$.
\red{Here or somewhere, mention notation as in Id? and follows WC01 as closely
as possible.... also note that these are formed as matrices but we use operators}

In this work, our focus is to develop a covariance model that can be used to
specify $\priorCovMat$.
In the general case, the control vector $\params$ could be multivariate,
including initial conditions of the system state, uncertain boundary
conditions, or uncertain parameter fields.
As such, the background-state covariance is
further decomposed so that a balance operator, $\balanceOperator$, is used
to handle the multivariate (i.e.\ cross-variable) covariance relationships
separately from the univariate (independent) covariances as
\begin{linenomath*}\begin{equation*}
    \priorCovMat \coloneqq \balanceOperator\unbalancedPriorCovMat\balanceOperator^T \,
    ,
\end{equation*}\end{linenomath*}
following \red{Derber and Bourtier, (1999)}.
The matrix $\unbalancedPriorCovMat$ describes the covariance for the unbalanced
variables and has a block-diagonal structure, such that each
covariance is described independently.
The unbalanced covariance is further factored as
\begin{linenomath*}\begin{equation*}
    \unbalancedPriorCovMat \coloneqq \Sigma \corrMat \Sigma
\end{equation*}\end{linenomath*}
where $\Sigma$ is a diagonal scaling matrix, containing the desired pointwise
standard deviation values and $\corrMat$ is a block diagonal correlation matrix,
describing each variable's independent correlation structure.
To be concrete, this can be viewed as
\begin{linenomath*}\begin{equation*}
    \unbalancedPriorCovMat =
    \begin{pmatrix}
        \Sigma_\alpha\corrMat_\alpha\Sigma_\alpha & 0 & \cdots & 0 \\
        0 & \Sigma_\beta\corrMat_\beta\Sigma_\beta & \cdots & 0 \\
        0 & 0 & \ddots & 0  \\
        0 & 0 & \cdots & \Sigma_\gamma\corrMat_\gamma\Sigma_\gamma \\
    \end{pmatrix}
\end{equation*}\end{linenomath*}
where $\alpha, \beta, \gamma$ are placeholders for unique variables, and each
$\Sigma_\alpha, \corrMat_\alpha$ pair describes the amplitude of pointwise standard deviation
and spatial correlation for each variable.

In this work, we present a methodology by which one can generically specify $\corrMat_\alpha$
with anisotropic and nonstationary correlations encoded into the field $\alpha$.
\red{\textbf{What to do next?}}
\begin{itemize}
    \item Talk about WC01 operator? and then get into why we want to do what we
        do?
    \item or go straight into Mat\'ern?
\end{itemize}



\subsection{Old intro}
A fundamental requirement for the inverse problem that we address in this work
is the specification of a prior distribution, $\priorDist(\params)$.
Recall from chapter XX that the multivariate control vector
$\params$ consists of the
potential temperature, salinity, and zonal velocity at the western boundary of
the domain:
$\params \coloneqq [\thetaParams^T,\saltParams^T,\uvelParams^T]^T \in\paramSpace$.
As is common in large scale geophysical inverse problems, we specify the prior
distribution for the control vector
to be Gaussian: $\priorDist(\params) \coloneqq \mathcal{N}(\params_0, \priorCovMat)$.

Our first objective in the inverse problem is to specify the prior
covariance $\priorCovMat$.
To do this, we obtain univariate prior covariances for
each individual field: $\thetaPriorCovMat$, $\saltPriorCovMat$,
$\uvelPriorCovMat$, which are then stacked block-diagonally to form
$\priorCovMat$ (see below for details).
Our primary focus in this chapter is to describe the generic formulation for
each of these univariate covariance matrices.
To facilitate the discussion, we refer to a generic univariate control variable
$\uni(\x)$ (or $\unis$ upon discretization), which in our specific case is a
placeholder for temperature, salinity, and zonal velocity, and is similarly
applicable to other inverse problems.
We use the general methodology outlined here to specify a prior covariance in
chapter XX.

In oceanographic inverse problems, covariance models must address at least
these three issues.
\begin{enumerate}
    \item Irregular boundaries imposed by continents
    \item Anisotropy due to the shallow fluid-like nature of the ocean
    \item Multivariate control parameters
\end{enumerate}
It is common to use covariance models based on differential operators in order
to handle irregular boundaries, and the question is then how to address the
other two issues within a differential equation.

A common approach to specifying the prior covariance in oceanographic inverse
problems is based on a generalized
diffusion equation \citep{weaver_correlation_2001}.
In this chapter, we outline an alternative approach that brings some practical
advantages which are discussed in section \ref{sec:matern_discussion}.
%#that has a similar structure to the
%#this model, but employs a version of the differential
%#operator presented in \citet{RSSB:RSSB777}.
%The general methodology is as follows.
We define a differential operator that can be represented by the matrix $C$,
that follows the flexible development from \citet{RSSB:RSSB777}.
We note that the matrix form is used for convenience, but that matrices are
never explicitly formed.
We show that the operator $C$ specifies the covariance matrix $CC^T$, that
is almost identical to a Mat\'ern covariance aside from boundary affects
imposed by $C$.
We then augment this differential operator with a sequence of factors that
are suggested by \citet{weaver_correlation_2001}.
That is, we incorporate the sequence of operations:
$\Sigma X C $, where
\begin{linenomath*}\begin{equation*}
    X \coloneqq \text{diag}\left\{ 1/\hat{\sigma}_{i}\right\}_{i=1}^{N}
\end{equation*}\end{linenomath*}
is a
normalization matrix computed from the pointwise marginal variance at grid cell
$i$: $\hat{\sigma}^2_{i}$, and
\begin{linenomath*}\begin{equation*}
    \Sigma \coloneqq \text{diag}\left\{\sigma_\uni\right\}_{i=1}^{\nuni}
\end{equation*}\end{linenomath*}
is the specified magnitude of prior uncertainty (standard deviation) for a
generic univariate parameter field $\unis\in\uniSpace$ (e.g.\ initial temperature).
With these definitions, $XCC^TX$ is a correlation matrix, and
\begin{linenomath}\begin{equation}
    \Gamma_\uni \coloneqq \Sigma X C C^T X \Sigma =
    \Gamma_\uni^{1/2}\Gamma_\uni^{T/2}
\end{equation}\end{linenomath}
defines the covariance for a generic univariate parameter field $\unis$.

The full prior covariance for a general multivariate application is then formed
by specifying individual covariance operators as above, and stacking them
together block diagonally.
To be concrete, the prior covariance for the specific inverse problem in this
work, i.e. for the steady state temperature, salinity,
and velocity fields at the open boundary of the computational domain,
is formulated as
\begin{linenomath*}\begin{equation*}
    \priorCovMat \coloneqq
    \begin{pmatrix}
        \thetaPriorCovMat & & \\
        & \saltPriorCovMat & \\
        & & \uvelPriorCovMat \\
    \end{pmatrix} \, .
\end{equation*}\end{linenomath*}
To keep this chapter general, we focus on specifying the covariance for a generic
univariate control parameter, $\uni(\x)$ which can be considered a placeholder
for each variable temperature, salinity, and velocity separately.
We wait until
chapter xx to discuss the specification of the prior
for each parameter, as this is becomes specific to our application.
Finally, we note that nonzero off diagonal terms in $\priorCovMat$ above (or an
additional operator) could be used to specify cross covariance between each
variable.
We consider this future work, and discuss potential options in
section XX.

In the following sections we review the general Mat\'ern type covariance
form that is suggested by \citet{RSSB:RSSB777}.
We then develop the differential operator, $C$, that
forms the backbone of our covariance model.
We show that our numerical implementation of the covariance model produces correlation
length scales that are expected from the analysis in section
\ref{sec:matern_operator}.
We conclude by discussing some advantages that this approach offers.


\section{Review of the Mat\'ern class covariance}
\label{sec:matern_review}

In this section we review the link between an elliptic stochastic partial
differential equation (SPDE) and Gaussian random fields.
The Mat\'ern covariance function between two points, $\xh_1,\xh_2\in\defdomain =
\ndspace$ can be expressed as:
\begin{linenomath}\begin{equation}
    c(\xh_1,\xh_2) = \dfrac{\sigma^2}{2^{\meandiff-1}
    \mathcal{G}(\meandiff)}
    \Big(\sqrt{\deltah} ||\xh_2-\xh_1||\Big)^\meandiff
    \mathcal{B}_\meandiff
    \Big(\sqrt{\deltah} ||\xh_2-\xh_1||\Big) \, .
    \label{eq:matern_covariance_iso}
\end{equation}\end{linenomath}
Here $||\cdot||$ as the Euclidean norm in $\defdomain$,
$\mathcal{G}$ is the Gamma function,
$\mathcal{B}_\meandiff$ is the modified
Bessel function of the second kind and order $\meandiff$,
$\sigma^2$ is the
marginal variance, $\deltah>0$ is a scaling parameter, and $\meandiff>0$
controls the mean-square differentiability of the underlying statistical process
described by the Mat\'ern covariance.
The reason for defining symbols with a hat ($\hat{\cdot}$), will become clear
in the next subsection.
Throughout, we refer to a ``Mat\'ern field'' as any Gaussian field that has
covariance that can be described by the Mat\'ern covariance function,
equation (\ref{eq:matern_covariance_iso}).

The key relationship discussed in \citet{RSSB:RSSB777} is that any Mat\'ern field,
$\unih(\xh)$, is a solution to the elliptic SPDE:
\begin{linenomath}\begin{equation}
    \Big(\deltah - \nablah\cdot\nablah\Big)^{\spdesqo/2}\hat{\uni}(\xh) =
    \Wh(\xh) \, .
    \label{eq:spde_iso}
\end{equation}\end{linenomath}
Here $\spdesqo = \meandiff + \materndim/2$,
$\Wh$ is a white noise process defined on the space $\defdomain$.
We note that the Mat\'ern covariance function describes covariances that are
stationary and isotropic.
That is, stationarity implies that correlation length scales are determined
purely by the Euclidean distance between two points and this does not change as
a function of location in the domain.
Isotropy implies that correlation lengths are the same for the same Euclidean
distance along any dimension.

The original connection between the Mat\'ern covariance function and solutions
to equation (\ref{eq:spde_iso}) was proven by
\cite{whittle_stationary_1954,whittle1963stochastic}, who
used the spectral properties of the operator $(\deltah -
\nablah\cdot\nablah)^{\spdesqo/2}$ to show that Mat\'ern fields are the only
stationary solutions to equation (\ref{eq:spde_iso}).
The result shown in \citet{RSSB:RSSB777} is
that there is an explicit link between discrete solutions to equation
(\ref{eq:spde_iso}) for any triangulation or rectangular lattice of $\ndspace$
and Mat\'ern class Gaussian fields.
The punch line is that we can use all of the computational tools for solving discretized
elliptic equations to apply a covariance operator that is formally dense.
More importantly, \citet{RSSB:RSSB777} showed that the SPDE form allows
one to easily describe Gaussian fields with more general covariance structures.
For instance, by allowing the parameter $\deltah$ to vary in space, the solution
becomes nonstationary and the Mat\'ern covariance applies locally.\\

\noindent\textbf{A more general covariance model. }
In \citet{RSSB:RSSB777} it is suggested that solving the SPDE in a transformed
coordinate system can allow for a Mat\'ern class covariance
model that can easily incorporate anisotropy and nonstationarity.
Consider the isotropic and stationary case, described by equation
(\ref{eq:spde_iso}).
The field $\unih(\xh)$ is defined in a transformed, or ``deformed''
\citep{sampson_nonparametric_1992}, space $\defdomain$.
Assume that we have a mapping $\defmap$ that maps between this transformed space
and our computational domain, $\domain$:
\begin{linenomath*}\begin{equation*}
    \defmap : \defdomain\ni\xh \rightarrow \x\in\domain \, .
\end{equation*}\end{linenomath*}
With this mapping, we can employ a change of variables
\citep{smith_change_1934} to rewrite the SPDE in the computational domain as:
\begin{linenomath*}\begin{equation*}
    \dfrac{1}{\defdet}
    \left(\deltah -
    \defdet\nabla\cdot
    \dfrac{\defjac(\x)\defjac(\x)^T}{\defdet}
    \nabla\right)\uni(\x) =
    \defdet^{-1/2}\W(\x) \, .
\end{equation*}\end{linenomath*}
Here we have defined the Jacobian as
\begin{linenomath*}\begin{equation*}
    \defjac(\x_0) \coloneqq
    \dfrac{\partial \defmap}{\partial \xh}\Big|_{\defmap^{-1}(\x_{0})} \, ,
\end{equation*}\end{linenomath*}
and for now we assume that $\defmap^{-1}(\x_0)$ is well defined.
For our purposes, this turns out to be the case, but this becomes clear when
$\defjac$ is defined in section \ref{sec:matern_operator}.
Notice that we have taken the exponent $\spdesqo/2$ to be 1, avoiding
fractional or higher order operations for simplicity.
All of the future formulations and experiments will make this assumption,
although this can be relaxed in future work.
With the following definitions:
\begin{linenomath}\begin{equation}
    K(\x) \coloneqq
    \dfrac{\defjac(\x)\defjac(\x)^T}{\defdet}
    \qquad
    \delta(\x) \coloneqq \dfrac{\deltah}{\defdet}
    \label{eq:matern_definitions}
\end{equation}\end{linenomath}
the SPDE in the computational domain's coordinate system can be written as
\begin{linenomath}\begin{equation}
    \Big(\delta(\x)- \nabla\cdot K(\x)\nabla\Big)\uni(\x) =
    \defdet^{-1/2}\W(\x) \, .
    \label{eq:spde_general}
\end{equation}\end{linenomath}
We note as in \citet{RSSB:RSSB777} that this reproduces the deformation method
introduced in \citet{sampson_nonparametric_1992}.
The key feature of this formulation is that the deformation map, $\defmap$, is
not actually necessary, only its Jacobian.
The question then becomes, how does one specify $\defjac$ and $\deltah$?
This is the primary question we wish to address.
However, we first develop the discretized form of this SPDE
that is relevant to the finite volume grid of our computational model, the
MITgcm.
