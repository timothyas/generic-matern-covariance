\section{Review of the Mat\'ern class covariance}
\label{sec:matern_review}

In this section we review the link between an elliptic stochastic partial
differential equation (SPDE) and Gaussian random fields.
The Mat\'ern covariance function between two points, $\xh_1,\xh_2\in\defdomain =
\ndspace$ can be expressed as:
\begin{equation}
    c(\xh_1,\xh_2) = \dfrac{\sigma^2}{2^{\meandiff-1}
    \mathcal{G}(\meandiff)}
    \Big(\sqrt{\deltah} ||\xh_2-\xh_1||\Big)^\meandiff
    \mathcal{B}_\meandiff
    \Big(\sqrt{\deltah} ||\xh_2-\xh_1||\Big) \, .
    \label{eq:matern_covariance_iso}
\end{equation}
Here $||\cdot||$ as the Euclidean norm in $\defdomain$,
$\mathcal{G}$ is the Gamma function,
$\mathcal{B}_\meandiff$ is the modified
Bessel function of the second kind and order $\meandiff$,
$\sigma^2$ is the
marginal variance, $\deltah>0$ is a scaling parameter, and $\meandiff>0$
controls the mean-square differentiability of the underlying statistical process
described by the Mat\'ern covariance.
The reason for defining symbols with a hat ($\hat{\cdot}$), will become clear
in the next subsection.
Throughout, we refer to a ``Mat\'ern field'' as any Gaussian field that has
covariance that can be described by the Mat\'ern covariance function,
equation (\ref{eq:matern_covariance_iso}).

The key relationship discussed in \citet{RSSB:RSSB777} is that any Mat\'ern field,
$\unih(\xh)$, is a solution to the elliptic SPDE:
\begin{equation}
    \Big(\deltah - \nablah\cdot\nablah\Big)^{\spdesqo/2}\hat{\uni}(\xh) =
    \Wh(\xh) \, .
    \label{eq:spde_iso}
\end{equation}
Here $\spdesqo = \meandiff + \materndim/2$,
$\Wh$ is a white noise process defined on the space $\defdomain$.
We note that the Mat\'ern covariance function describes covariances that are
stationary and isotropic.
That is, stationarity implies that correlation length scales are determined
purely by the Euclidean distance between two points and this does not change as
a function of location in the domain.
Isotropy implies that correlation lengths are the same for the same Euclidean
distance along any dimension.

The original connection between the Mat\'ern covariance function and solutions
to equation (\ref{eq:spde_iso}) was proven by
\cite{whittle_stationary_1954,whittle1963stochastic}, who
used the spectral properties of the operator $(\deltah -
\nablah\cdot\nablah)^{\spdesqo/2}$ to show that Mat\'ern fields are the only
stationary solutions to equation (\ref{eq:spde_iso}).
The result shown in \citet{RSSB:RSSB777} is
that there is an explicit link between discrete solutions to equation
(\ref{eq:spde_iso}) for any triangulation or rectangular lattice of $\ndspace$
and Mat\'ern class Gaussian fields.
The punch line is that we can use all of the computational tools for solving discretized
elliptic equations to apply a covariance operator that is formally dense.
More importantly, \citet{RSSB:RSSB777} showed that the SPDE form allows
one to easily describe Gaussian fields with more general covariance structures.
For instance, by allowing the parameter $\deltah$ to vary in space, the solution
becomes nonstationary and the Mat\'ern covariance applies locally.\\

\noindent\textbf{A more general covariance model. }
In \citet{RSSB:RSSB777} it is suggested that solving the SPDE in a transformed
coordinate system can allow for a Mat\'ern class covariance
model that can easily incorporate anisotropy and nonstationarity.
Consider the isotropic and stationary case, described by equation
(\ref{eq:spde_iso}).
The field $\unih(\xh)$ is defined in a transformed, or ``deformed''
\citep{sampson_nonparametric_1992}, space $\defdomain$.
Assume that we have a mapping $\defmap$ that maps between this transformed space
and our computational domain, $\domain$:
\begin{equation*}
    \defmap : \defdomain\ni\xh \rightarrow \x\in\domain \, .
\end{equation*}
With this mapping, we can employ a change of variables
\citep{smith_change_1934} to rewrite the SPDE in the computational domain as:
\begin{equation*}
    \dfrac{1}{\defdet}
    \left(\deltah -
    \defdet\nabla\cdot
    \dfrac{\defjac(\x)\defjac(\x)^T}{\defdet}
    \nabla\right)\uni(\x) =
    \defdet^{-1/2}\W(\x) \, .
\end{equation*}
Here we have defined the Jacobian as
\begin{equation*}
    \defjac(\x_0) \coloneqq
    \dfrac{\partial \defmap}{\partial \xh}\Big|_{\defmap^{-1}(\x_{0})} \, ,
\end{equation*}
and for now we assume that $\defmap^{-1}(\x_0)$ is well defined.
For our purposes, this turns out to be the case, but this becomes clear when
$\defjac$ is defined in section \ref{sec:matern_operator}.
Notice that we have taken the exponent $\spdesqo/2$ to be 1, avoiding
fractional or higher order operations for simplicity.
All of the future formulations and experiments will make this assumption,
although this can be relaxed in future work.
With the following definitions:
\begin{equation}
    K(\x) \coloneqq
    \dfrac{\defjac(\x)\defjac(\x)^T}{\defdet}
    \qquad
    \delta(\x) \coloneqq \dfrac{\deltah}{\defdet}
    \label{eq:matern_definitions}
\end{equation}
the SPDE in the computational domain's coordinate system can be written as
\begin{equation}
    \Big(\delta(\x)- \nabla\cdot K(\x)\nabla\Big)\uni(\x) =
    \defdet^{-1/2}\W(\x) \, .
    \label{eq:spde_general}
\end{equation}
We note as in \citet{RSSB:RSSB777} that this reproduces the deformation method
introduced in \citet{sampson_nonparametric_1992}.
The key feature of this formulation is that the deformation map, $\defmap$, is
not actually necessary, only its Jacobian.
The question then becomes, how does one specify $\defjac$ and $\deltah$?
This is the primary question we wish to address.
However, we first develop the discretized form of this SPDE
that is relevant to the finite volume grid of our computational model, the
MITgcm.
