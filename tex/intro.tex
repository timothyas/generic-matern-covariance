\section{Introduction}
\label{sec:intro}

Large-scale geophysical inverse problems,
such as numerical weather prediction or
oceanographic state estimation, are typically ill-posed due to the
sparsity of data relative to the size of the informed state.
A classical method for handling this ill-posedness is to prescribe
some estimate of the background-state error, capturing our degree of confidence in the
estimated control vector at a given point in space and time.
For all practical applications, true knowledge of the background-state
uncertainty must be approximated and, to make the problem computationally
tractable, it is often assumed to be governed by Gaussian statistics.
Under this assumption, the uncertainty is fully described by a covariance
matrix that is often represented by a parameterized model.

For applications with simple geometries, covariance models can be specified with
simple, closed form smoothing (e.g.\ exponential) functions.
\red{some definition/development of isotropic/stationary, why aniso/nonstat
necessary, etc}.
In numerical weather prediction,
covariance relationships are typically represented in terms of a spherical
harmonic expansion \red{CITE}.
However, alternative approaches are required for geophysical systems that have
more complicated geometries, such as in oceanography
where irregular continental boundaries make these techniques less practical.
In such applications, it is advantageous to pose the background-state covariance
model in terms of the original geophysical coordinate, and employ models based on differential
equations that maintain some correspondence to statistical covariance.

In oceanographic state estimation, the most widely used static background-state
covariance model is based on the work by
\citet[][WC01 hereafter]{weaver_correlation_2001},
\citep[e.g.][]{nguyen_arctic_2021,forgetECCOv4,moore_regional_2011-1,mazloff_eddy-permitting_2010,gebbie_strategies_2006}.
Briefly, the WC01 model employs a diffusion-based operator to establish
correlation length scales, which can be rescaled to produce the desired covariance
information (see Section \ref{ssec:wc01_review} for a more detailed review).
Generally speaking, the model is advantageous because
the diffusion operator at its core scales well to high dimensional inverse problems
(e.g. $>10^8$ degrees of freedom in \citet{forgetECCOv4}).
Additionally, irregular continental boundaries are handled naturally by
supplying boundary conditions to the differential equation - typically zero
flux.

% Drawbacks
%\begin{itemize}
%   Actually, this isn't generally true ... it's just that this is the way that
%   most people implement WC01
%   See WC01 section 5d
%    \item To achieve anisotropy due to vastly different correlation
%        length-scales in the horizontal compared to the vertical, requires the
%        assumption of separability. This seems to work in practice even though
%        there is no true justification for it, but comes at the computational
%        expense of applying
%        solving an explicit diffusion equation twice - once for the horizontal
%        and again for the vertical.
%
%   Is it worth it to point this one out on its own?
%    \item The number of pseudo time steps required for this diffusion solve to
%        be numerically stable requires some trial and error. In WC01, the
%        authors provide a minimum threshold, but in practice this is not
%        sufficiently long enough, requiring some trial and error to determine
%        the appropriate integration time.
%\end{itemize}

% So what's new here? Background.
More recently, \citet{RSSB:RSSB777} showed an alternative approach for
prescribing Mat\'ern-type Gaussian covariance relations in random fields.
The approach is highly scalable and applicable to computational domains with
irregular boundaries for similar reasons as for the WC01 model -
it is based on the solution to an elliptic Partial
Differential Equation (PDE).
Moreover, \citet{RSSB:RSSB777} allude to a mapping method that would allow the
covariance model to achieve anisotropic and nonstationary relations.
As of yet, however, there has been no development or implementation of this
mapping method to geophysical inverse problems.

Here, we show how the mapping method introduced by \citet{RSSB:RSSB777}
can be parameterized based on the finite volume method
in order to achieve anisotropic and nonstationary relationships, such that it can easily
be deployed in many commonly used ocean general circulation models.
At the core of our development is a range parameter, which carries the intuitive
interpretation as the number of neighboring grid cells at which
correlation decays to 0.1.
Our numerical experiments show that correlations derived from the isotropic,
stationary Mat\'ern correlation function can be mapped to a generic ocean model
grid in terms of this range parameter.
It is therefore straightforward to use this approach to achieve desirable
anisotropic and nonstationary statistics by tuning a single parameter that
corresponds to an ocean model grid.
%Our numerical experiments show how one can use this parameter together with the
%isotropic, stationary Mat\'ern correlation function to predict and achieve desirable
%statistics in the more general case.
%Moreover, we show that the elliptic PDE at the heart of the covariance model can
%be solved with a high ($\sim\bigo(10^{-2})$) tolerance while still achieving the
%desired correlation statistics, such that it is very efficient.
%\red{Discuss how this is advantageous to WC01 guess work?}

% Is this really necessary?
The paper is laid out as follows.
In \cref{sec:review} we give some context for how the covariance model is
used in data assimilation or inverse problems.
We provide a general overview of the covariance model developed in WC01, and
then review the Mat\'ern type covariance developments in \citet{RSSB:RSSB777}
which we build on.
In \cref{sec:discretization_matern} we show the discretized form of the covariance model
that is relevant to finite volume discretization schemes common to ocean
general circulation models.
We then describe how the covariance model parameters can be specified to achieve
an intuitive relationship between the isotropic, stationary form and the more
general anisotropic, nonstationary form.
In \cref{sec:llc90} we show numerical results of the nonstationary, anisotropic
correlation model applied to the global ocean, using the ``Lat-Lon-Cap''
grid introduced by \citet{forgetECCOv4}.
Finally, in \cref{sec:matern_discussion} we show how our methodology compares to the
commonly used WC01 covariance model, and conclude with a discussion on the subtle numerical
advantages of this approach.
