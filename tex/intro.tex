\section{Introduction}
\label{sec:intro}

High dimensional geophysical inverse problems,
such as numerical weather prediction and
oceanographic state estimation, are typically ill-posed due to the
sparsity of data relative to the size of the control vector.
A classical method for handling this ill-posedness is to prescribe
some type of regularization in order to ``spread'' information to the uninformed
regions and variables in the control vector \citep[e.g.,][]{wunsch_discrete_2006}.
In the Bayesian interpretation of the inverse problem, this regularization
is defined so that it represents the prior uncertainty or background-state error
\citep[e.g.,][]{bui-thanh_computational_2013}.
Ideally, this uncertainty captures the true error in the background state, but
for all practical applications the background error must be
approximated.
Moreover, to make the problem computationally tractable, it is often assumed
that the background error is governed by Gaussian statistics, such that the
uncertainty is fully described by a covariance matrix.

For realistic data assimilation (DA) problems in meteorology and
oceanography, a well formed background error will contain covariance relationships between
different variables (e.g., between temperature and velocity components),
it will have spatially dependent length scales of covariation
(i.e.\ nonstationarity or inhomogeneity),
and it will respect the system's anisotropy, such that length scales of covariance differ
appropriately in each direction \citep[e.g.,][]{bannister_review_2008-1}.
Within variational DA systems, the background error covariance is
usually represented as an operator so that it can be applied efficiently during
an iterative optimization algorithm.
Thus, for an operator-based covariance model to be useful in this context, it must be
able to respect multivariate, nonstationary, and anisotropic features that are
necessary for the given problem setting.

Typically, the background state error covariance is decomposed into two
operators following \citet{derber_reformulation_1999}.
The first, ``balance'' operator captures multivariate (i.e.\ cross-variable)
covariance information while the second captures ``unbalanced'' (i.e.\
univariate) covariance information.
Our focus is on the latter operator, but we note the review by
\citet{bannister_review_2008-2} which outlines balance operators used in
atmospheric DA, and e.g., \citet{weaver_multivariate_2005,moore_regional_2011-1} for oceanographic
examples.

Univariate covariance operators are further decomposed into two stages, where spatial
correlations are specified first, and then scaled to the appropriate amplitude.
In atmospheric DA, it is common to transform the control vector
into a wavelet or spherical harmonic basis in order to specify the background
error \citep[e.g.,][]{bannister_review_2008-2}.
However, for systems with complex boundaries like the ocean, it is far more
straightforward to formulate the correlation operator in terms of the original,
physical domain.
Considering methods that operate in the physical or grid space, there are
generally two classes of correlation models that are commonly used.
The first class is encapsulated by explicit functional forms, including, for
example, the functions developed by
\citet{gaspari_construction_1999,gneiting_correlation_1999,gaspari_construction_2006},
which have the benefit of providing compact support.
The second class of correlation models can generally be described as a filtering
approach.
\citet{purser_numerical_2003-2,purser_numerical_2003-1}
show correlation functions based on recursive filters, and
\citet{dobricic_oceanographic_2008} extend this to be used with complex boundaries.
More recently, \citet{purser_multigrid_2022} show a ``beta
filter'' approach which enables compact support and a highly generalizable
correlation shape via an efficient multigrid approach.
Alternatively, within this class of models are those based on the solution of a
differential equation.

Correlation models that are based on the solution to differential operators have
several advantages.
Most importantly, these operators handle complex boundaries naturally and the
infrastructure required to obtain their solution typically exist within
the underlying numerical model.
Within oceanographic state estimation, a widely used framework is based on the
solution to the diffusion equation
\citep[e.g.,][]{nguyen_arctic_2021,forgetECCOv4,blockley_recent_2014,moore_regional_2011-1,daget_ensemble_2009,muccino_inverse_2008,di_lorenzo_weak_2007,weaver_three-_2003}.
The diffusion-based framework relies on either an explicit, pseudo-time stepping
method \citep{weaver_correlation_2001} or an implicit solution
\citep{mirouze_representation_2010,carrier_background-error_2010,weaver_diffusion_2013},
where the correlation structure underlying the solution corresponds to either a
Gaussian or more general auto-regressive function, respectively.
Alternatively, \citet{RSSB:RSSB777} show an explicit link between the numerical solution
to a stochastic elliptic partial differential equation and a Mat\'ern-type covariance
model.
As of yet, however, it has remained unclear how this model could be used to
specify univariate correlations with appropriate nonstationarity and anisotropy
for an operational variational DA system.

Here, we extend the work by \citet{RSSB:RSSB777}
to show how the framework can be used within variational DA.
Our emphasis is on oceanographic applications, although the methodology is more
general.
We show how the mapping method introduced by \citet{RSSB:RSSB777}
can be used to formulate a correlation operator that respects anisotropy and
nonstationarity in a way that is relevant to many ocean general circulation
models.
At the core of this model there are two parameters, which \textit{separately}
control the shape and length scale of the correlation model.
We show that the parameter controlling the length scale can be interpreted
intuitively as the ``number of neighboring grid cells'' at which correlation
drops to an expected value.
Our numerical experiments show that correlations derived from the isotropic,
stationary Mat\'ern correlation function can be mapped to a generic ocean model
grid in terms of this range parameter.
It is therefore straightforward to use this approach to achieve desirable
anisotropic and nonstationary statistics by simply tuning these two
parameters.

% Is this really necessary?
The paper is laid out as follows.
In \cref{sec:review} we give some context for how univariate correlation models
are used in variational data assimilation.
We provide a review of diffusion based correlation operators, and
then review the Mat\'ern type covariance developments in \citet{RSSB:RSSB777}
which we build on.
In \cref{sec:matern_operator} we show how the Mat\'ern model can be mapped from
its isotropic, stationary form into a more complex computational domain.
We then provide suggestions for parameterizing the model so that it can
intuitively capture anisotropy and nonstationarity.
In \cref{sec:llc90} we show numerical results of this correlation model
applied to the global ocean, using the ``Lat-Lon-Cap'' grid introduced by
\citet{forgetECCOv4}.
Finally, in \cref{sec:matern_discussion} we provide some discussion on the
advantages of this model, and a general comparison to the widely used
diffusion based models.
