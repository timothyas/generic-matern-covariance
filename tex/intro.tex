\section{Introduction}
\label{sec:intro}

Geophysical inverse problems, such as numerical weather prediction \red{cite} or
oceanographic state estimation \red{cite}, are typically ill-posed due to the
sparsity of data relative to the size of the control vector.
A classical method for handling this ill-posedness is to prescribe some estimate
of the background-state error, capturing our degree of confidence in the
estimated control vector at a given point in space and time \red{cite}.
In deterministic inverse problems, this is referred to as regularization and
corresponds to prior information in the probabilistic sense.
For all practical applications, true knowledge of the background state error
must be approximated.
A common approach is to prescribe a static covariance, where standard
deviations and length-scales that describe the covariance are captured by
various parameters.
\red{Dynamic error covariance?}

For applications with simple geometries, covariance models can be specified with
simple exponential functions.
However, alternative approaches are required for more complicated domains,
such as oceanographic state estimation,
where irregular continental boundaries impede the use of spherical harmonics,
which are commonly used in atmospheric inverse problems, or other analytical
functions.
In such applications, it is advantageous to cast the background-state covariance
in terms of geophysical coordinates and employ models based on differential
equations which have some correspondence to covariance information.
\red{More overview}

In oceanographic state estimation, the most widely used static background-state
covariance model is based on the work by
\citet[][WC01 hereafter]{weaver_correlation_2001},
\citep[e.g.][]{nguyen_arctic_2021,forgetECCOv4,moore_regional_2011-1,mazloff_eddy-permitting_2010,gebbie_strategies_2006}.
At its core, the WC01 model employs a diffusion-based operator to achieve
correlation length scales, which are rescaled to produce the desired covariance
information (see Section XX for a more detailed review).
Generally speaking, the model is advantageous because
the diffusion operator scales well to high dimensional inverse problems
(e.g. $>10^7$ degrees of freedom in \citet{forgetECCOv4}).
Additionally, irregular continental boundaries are handled naturally by
supplying boundary conditions to the differential equation - typically zero
flux.

% Drawbacks
%\begin{itemize}
%   Actually, this isn't generally true ... it's just that this is the way that
%   most people implement WC01
%   See WC01 section 5d
%    \item To achieve anisotropy due to vastly different correlation
%        length-scales in the horizontal compared to the vertical, requires the
%        assumption of separability. This seems to work in practice even though
%        there is no true justification for it, but comes at the computational
%        expense of applying
%        solving an explicit diffusion equation twice - once for the horizontal
%        and again for the vertical.
%
%   Is it worth it to point this one out on its own?
%    \item The number of pseudo time steps required for this diffusion solve to
%        be numerically stable requires some trial and error. In WC01, the
%        authors provide a minimum threshold, but in practice this is not
%        sufficiently long enough, requiring some trial and error to determine
%        the appropriate integration time.
%\end{itemize}

% So what's new here? Background.
More recently, \citet{RSSB:RSSB777} showed an alternative approach for
prescribing Mat\'ern-type Gaussian covariance relations in random fields.
The approach is highly scalable and applicable to computational domains with
irregular boundaries for similar reasons as for the WC01 model -
it is based on the solution to an elliptic Partial
Differential Equation (PDE).
Moreover, \citet{RSSB:RSSB777} allude to a mapping method that would allow the
covariance model to achieve anisotropic and nonstationary relations.
As of yet, however, there has been no development or implementation of this
mapping method to geophysical inverse problems.

Here, we show how the mapping method introduced by \citet{RSSB:RSSB777}
can be parameterized based on the underlying discretization
in order to achieve anisotropic and nonstationary relationships.
At the core of our development is a range parameter, which can be intuitively
interpreted as an indication of the number of neighboring grid cells at which
correlation decays.
Our numerical experiments show how one can use this parameter together with the
isotropic, stationary Mat\'ern correlation function to predict and achieve desirable
statistics in the more general case.
Moreover, we show that the elliptic PDE at the heart of the covariance model can
be solved with a high ($\sim\bigo(10^{-2})$) tolerance while still achieving the
desired correlation statistics, such that it is very efficient.
\red{Discuss how this is advantageous to WC01 guess work?}

% Is this really necessary?
The paper is laid out as follows.
In \cref{sec:matern_review} we give some context for how the covariance model is
used in data assimilation or inverse problems.
We provide a general overview of the covariance model developed in WC01, and
then review the Mat\'ern type covariance developments in \citet{RSSB:RSSB777}
which we build on.
In \cref{sec:matern_operator} we show the discretized form of the covariance model as
relevant to finite volume discretization schemes that are common to ocean
general circulation models.
We then describe how the covariance model parameters can be specified to achieve
an intuitive relationship between the isotropic, stationary form and the more
general anisotropic, nonstationary form.
In \cref{sec:llc90} we show numerical results of the nonstationary, anisotropic
correlation model applied to the global ocean, using the ``Lat-Lon-Cap''
discretization (topology?) from \citet{forgetECCOv4}.
The numerically derived correlations compare well to our theoretical
expectations
Finally, in \cref{sec:matern_discussion} we compare to WC01, and provide
conclusions in X.
