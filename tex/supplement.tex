\documentclass[alpha-refs]{Wiley-LaTeX-Template/wiley-article}

\papertype{Supplemental Material}

% --- Journal Provided
\usepackage{url}
\usepackage{lineno}
\usepackage{soul}

% --- Custom additions
\usepackage{amsmath, amssymb}
\usepackage{mathtools} % for \coloneqq
\usepackage{color}
\usepackage{graphicx}
\usepackage{natbib}
\usepackage{bm}
\usepackage{adjustbox}
\usepackage[percent]{overpic}

% Definitions
\usepackage{ide}

\usepackage{hyperref}
\hypersetup{
    colorlinks=true,
    linkcolor=black,
    filecolor=black,
    urlcolor=black,
    citecolor=black
}
\usepackage[capitalise,noabbrev]{cleveref}

\renewcommand{\thefigure}{S\arabic{figure}}

% --- Header
\title{Supplemental Material for:
    ``A Practical Formulation for an Anisotropic and Nonstationary Mat\'ern Class
    Correlation Operator''
}
\author{Timothy A. Smith}

\runningauthor{Smith}

\begin{document}

\maketitle

\newpage
\begin{figure}
    \centering
    \begin{overpic}[width=\textwidth]{../figures/matern_llc90_correlation_theory_vs_m_iy.pdf}
        \put(6,35){(a)}
        \put(30,35){(b)}
        \put(53,35){(c)}
        \put(76,35){(d)}
    \end{overpic}
    \caption{Correlation structure computed from the theoretical Mat\'ern
        correlation function (black) and from
        1,000 samples using a subset of the ``Lat-Lon-Cap'' grid within the
        Pacific Ocean (shaded coloring).
        The sample correlation is computed in the meridional direction, $\delta j$,
        indicating the number of neighboring grid cells from 0.2$^\circ$N.
        The shading indicates the spread between the first and ninth deciles,
        based on sample correlations at all depth levels and latitudes from
        157.5$^\circ$W and 97.5$^\circ$W.
    }
    \label{fig:llc90_correlations_j}
\end{figure}

\begin{figure}
    \centering
    \begin{overpic}[width=\textwidth]{../figures/matern_llc90_correlation_theory_vs_m_k.pdf}
        \put(6,35){(a)}
        \put(30,35){(b)}
        \put(53,35){(c)}
        \put(76,35){(d)}
    \end{overpic}
    \caption{Same as \cref{fig:llc90_correlations_j}, but
        the sample correlation is computed in the vertical direction, $\delta k$,
        indicating the number of neighboring grid cells from 722~m depth (the
        25th vertical level).
        The shading indicates the spread between the first and ninth deciles,
        based on sample correlations at each grid cell from
        157.5$^\circ$W to 97.5$^\circ$W in the zonal direction and
        70$^\circ$S to 37$^\circ$N in the meridional direction.
        There are 50 vertical levels on the LLC grid.
    }
    \label{fig:llc90_correlations_k}
\end{figure}

\end{document}
